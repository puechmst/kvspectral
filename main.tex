%  LaTeX support: latex@mdpi.com 
%  For support, please attach all files needed for compiling as well as the log file, and specify your operating system, LaTeX version, and LaTeX editor.

%=================================================================
\documentclass[mathematics,article,submit,pdftex,moreauthors]{Definitions/mdpi} 

\usepackage{amssymb,amsmath,amsfonts,amsthm}
\usepackage{graphicx}
\usepackage{stmaryrd}
\usepackage[utf8]{inputenc}
\usepackage[T1]{fontenc}
\usepackage{tikz-cd}

\newcommand{\R}{\ensuremath{\mathbb{R}}}
\newcommand{\C}{\ensuremath{\mathbb{C}}}
\newcommand{\K}{\ensuremath{\mathbb{K}}}
\newcommand{\Q}{\ensuremath{\mathbb{Q}}}
\newcommand{\N}{\ensuremath{\mathbb{N}}}
\newcommand{\Z}{\ensuremath{\mathbb{Z}}}
\newcommand{\usph}[1]{\ensuremath{\mathbb{S}^{#1}}}
\newcommand{\Aff}{\ensuremath{\text{Aff}}}
\newcommand{\aff}{\ensuremath{\mathfrak{aff}}}
\newcommand{\gl}{\ensuremath{\mathfrak{gl}}}
\newcommand{\GL}{\ensuremath{\text{GL}}}
\newcommand{\homfunc}[2]{\ensuremath{\text{Hom}\left(#1,#2\right)}}
\newcommand{\comp}[2]{\ensuremath{\text{H}\left(#1,#2\right)}}
\newcommand{\homology}[3]{\ensuremath{\text{H}^{#1}\left(#2,#3\right)}}
\newcommand{\frakg}{\ensuremath{\mathfrak{g}}}
\newcommand{\pairc}[2]{\ensuremath{\left\langle #1,#2 \right\rangle_+}}
\newcommand{\paircs}[2]{\ensuremath{\left\langle #1,#2 \right\rangle_-}}
\newcommand{\pairct}[2]{\ensuremath{\left\langle #1,#2 \right\rangle_\theta}}
\newcommand{\cbracket}[2]{\ensuremath{\left\llbracket #1,#2 \right\rrbracket_c}}
\newcommand{\dbracket}[2]{\ensuremath{\left\llbracket #1,#2 \right\rrbracket_d}}
\newcommand{\lieder}[2]{\ensuremath{\mathcal{L}_#1 #2}}
\newcommand{\parder}[2]{\ensuremath{\frac{\partial #1}{\partial #2}}}
\newcommand{\gmet}[2]{\ensuremath{g\left(#1, #2  \right)}}
\newcommand{\pullg}[2]{\ensuremath{\tilde{g}\left(#1, #2  \right)}}
\newcommand{\nnet}[2]{\ensuremath{\mathcal{N}\left( #1,#2 \right)}}
\newcommand{\mnet}[1]{\ensuremath{\mathcal{N}_{W}\left( #1\right)}}
\newcommand{\lc}{\ensuremath{\nabla^{\text{lc}}}}
\newcommand{\fnet}{\ensuremath{\mathcal{N}_W}}
\newcommand{\kldiv}[2]{\ensuremath{\textit{KL}\left( #1,#2 \right)}}
\newcommand{\im}{\ensuremath{\textsf{im }}}
\DeclareMathOperator{\ad}{ad}
\DeclareMathOperator{\trace}{\ensuremath{\text{Tr}}}
\DeclareMathOperator{\ext}{\ensuremath{\text{Ext}}}

\theoremstyle{plain}
\newtheorem{thm}{Theorem}[section]
\newtheorem{lem}[thm]{Lemma}
\newtheorem{prop}[thm]{Proposition}
\newtheorem*{cor}{Corollary}

\theoremstyle{remark}
\newtheorem*{rem}{Remark}

\theoremstyle{definition}
\newtheorem{defn}{Definition}[section]

%--------------------
% Class Options:
%--------------------
%----------
% journal
%----------
% Choose between the following MDPI journals:
% acoustics, actuators, addictions, admsci, adolescents, aerobiology, aerospace, agriculture, agriengineering, agrochemicals, agronomy, ai, air, algorithms, allergies, alloys, analytica, analytics, anatomia, animals, antibiotics, antibodies, antioxidants, applbiosci, appliedchem, appliedmath, applmech, applmicrobiol, applnano, applsci, aquacj, architecture, arm, arthropoda, arts, asc, asi, astronomy, atmosphere, atoms, audiolres, automation, axioms, bacteria, batteries, bdcc, behavsci, beverages, biochem, bioengineering, biologics, biology, biomass, biomechanics, biomed, biomedicines, biomedinformatics, biomimetics, biomolecules, biophysica, biosensors, biotech, birds, bloods, blsf, brainsci, breath, buildings, businesses, cancers, carbon, cardiogenetics, catalysts, cells, ceramics, challenges, chemengineering, chemistry, chemosensors, chemproc, children, chips, cimb, civileng, cleantechnol, climate, clinpract, clockssleep, cmd, coasts, coatings, colloids, colorants, commodities, compounds, computation, computers, condensedmatter, conservation, constrmater, cosmetics, covid, crops, cryptography, crystals, csmf, ctn, curroncol, cyber, dairy, data, ddc, dentistry, dermato, dermatopathology, designs, devices, diabetology, diagnostics, dietetics, digital, disabilities, diseases, diversity, dna, drones, dynamics, earth, ebj, ecologies, econometrics, economies, education, ejihpe, electricity, electrochem, electronicmat, electronics, encyclopedia, endocrines, energies, eng, engproc, entomology, entropy, environments, environsciproc, epidemiologia, epigenomes, est, fermentation, fibers, fintech, fire, fishes, fluids, foods, forecasting, forensicsci, forests, foundations, fractalfract, fuels, future, futureinternet, futurepharmacol, futurephys, futuretransp, galaxies, games, gases, gastroent, gastrointestdisord, gels, genealogy, genes, geographies, geohazards, geomatics, geosciences, geotechnics, geriatrics, grasses, gucdd, hazardousmatters, healthcare, hearts, hemato, hematolrep, heritage, higheredu, highthroughput, histories, horticulturae, hospitals, humanities, humans, hydrobiology, hydrogen, hydrology, hygiene, idr, ijerph, ijfs, ijgi, ijms, ijns, ijpb, ijtm, ijtpp, ime, immuno, informatics, information, infrastructures, inorganics, insects, instruments, inventions, iot, j, jal, jcdd, jcm, jcp, jcs, jcto, jdb, jeta, jfb, jfmk, jimaging, jintelligence, jlpea, jmmp, jmp, jmse, jne, jnt, jof, joitmc, jor, journalmedia, jox, jpm, jrfm, jsan, jtaer, jvd, jzbg, kidneydial, kinasesphosphatases, knowledge, land, languages, laws, life, liquids, literature, livers, logics, logistics, lubricants, lymphatics, machines, macromol, magnetism, magnetochemistry, make, marinedrugs, materials, materproc, mathematics, mca, measurements, medicina, medicines, medsci, membranes, merits, metabolites, metals, meteorology, methane, metrology, micro, microarrays, microbiolres, micromachines, microorganisms, microplastics, minerals, mining, modelling, molbank, molecules, mps, msf, mti, muscles, nanoenergyadv, nanomanufacturing,\gdef\@continuouspages{yes}} nanomaterials, ncrna, ndt, network, neuroglia, neurolint, neurosci, nitrogen, notspecified, %%nri, nursrep, nutraceuticals, nutrients, obesities, oceans, ohbm, onco, %oncopathology, optics, oral, organics, organoids, osteology, oxygen, parasites, parasitologia, particles, pathogens, pathophysiology, pediatrrep, pharmaceuticals, pharmaceutics, pharmacoepidemiology,\gdef\@ISSN{2813-0618}\gdef\@continuous pharmacy, philosophies, photochem, photonics, phycology, physchem, physics, physiologia, plants, plasma, platforms, pollutants, polymers, polysaccharides, poultry, powders, preprints, proceedings, processes, prosthesis, proteomes, psf, psych, psychiatryint, psychoactives, publications, quantumrep, quaternary, qubs, radiation, reactions, receptors, recycling, regeneration, religions, remotesensing, reports, reprodmed, resources, rheumato, risks, robotics, ruminants, safety, sci, scipharm, sclerosis, seeds, sensors, separations, sexes, signals, sinusitis, skins, smartcities, sna, societies, socsci, software, soilsystems, solar, solids, spectroscj, sports, standards, stats, std, stresses, surfaces, surgeries, suschem, sustainability, symmetry, synbio, systems, targets, taxonomy, technologies, telecom, test, textiles, thalassrep, thermo, tomography, tourismhosp, toxics, toxins, transplantology, transportation, traumacare, traumas, tropicalmed, universe, urbansci, uro, vaccines, vehicles, venereology, vetsci, vibration, virtualworlds, viruses, vision, waste, water, wem, wevj, wind, women, world, youth, zoonoticdis 
% For posting an early version of this manuscript as a preprint, you may use "preprints" as the journal. Changing "submit" to "accept" before posting will remove line numbers.

%---------
% article
%---------
% The default type of manuscript is "article", but can be replaced by: 
% abstract, addendum, article, book, bookreview, briefreport, casereport, comment, commentary, communication, conferenceproceedings, correction, conferencereport, entry, expressionofconcern, extendedabstract, datadescriptor, editorial, essay, erratum, hypothesis, interestingimage, obituary, opinion, projectreport, reply, retraction, review, perspective, protocol, shortnote, studyprotocol, systematicreview, supfile, technicalnote, viewpoint, guidelines, registeredreport, tutorial
% supfile = supplementary materials

%----------
% submit
%----------
% The class option "submit" will be changed to "accept" by the Editorial Office when the paper is accepted. This will only make changes to the frontpage (e.g., the logo of the journal will get visible), the headings, and the copyright information. Also, line numbering will be removed. Journal info and pagination for accepted papers will also be assigned by the Editorial Office.

%------------------
% moreauthors
%------------------
% If there is only one author the class option oneauthor should be used. Otherwise use the class option moreauthors.

%---------
% pdftex
%---------
% The option pdftex is for use with pdfLaTeX. Remove "pdftex" for (1) compiling with LaTeX & dvi2pdf (if eps figures are used) or for (2) compiling with XeLaTeX.

%=================================================================
% MDPI internal commands - do not modify
\firstpage{1} 
\makeatletter 
\setcounter{page}{\@firstpage} 
\makeatother
\pubvolume{1}
\issuenum{1}
\articlenumber{0}
\pubyear{2024}
\copyrightyear{2024}
%\externaleditor{Academic Editor: Firstname Lastname}
\datereceived{ } 
\daterevised{ } % Comment out if no revised date
\dateaccepted{ } 
\datepublished{ } 
%\datecorrected{} % For corrected papers: "Corrected: XXX" date in the original paper.
%\dateretracted{} % For corrected papers: "Retracted: XXX" date in the original paper.
\hreflink{https://doi.org/} % If needed use \linebreak
%\doinum{}
%\pdfoutput=1 % Uncommented for upload to arXiv.org
%\CorrStatement{yes}  % For updates


%=================================================================
% Add packages and commands here. The following packages are loaded in our class file: fontenc, inputenc, calc, indentfirst, fancyhdr, graphicx, epstopdf, lastpage, ifthen, float, amsmath, amssymb, lineno, setspace, enumitem, mathpazo, booktabs, titlesec, etoolbox, tabto, xcolor, colortbl, soul, multirow, microtype, tikz, totcount, changepage, attrib, upgreek, array, tabularx, pbox, ragged2e, tocloft, marginnote, marginfix, enotez, amsthm, natbib, hyperref, cleveref, scrextend, url, geometry, newfloat, caption, draftwatermark, seqsplit
% cleveref: load \crefname definitions after \begin{document}

%=================================================================
% Please use the following mathematics environments: Theorem, Lemma, Corollary, Proposition, Characterization, Property, Problem, Example, ExamplesandDefinitions, Hypothesis, Remark, Definition, Notation, Assumption
%% For proofs, please use the proof environment (the amsthm package is loaded by the MDPI class).

%=================================================================
% Full title of the paper (Capitalized)
\Title{Title}

% MDPI internal command: Title for citation in the left column
\TitleCitation{Title}

% Author Orchid ID: enter ID or remove command
\newcommand{\orcidauthorA}{0000-0000-0000-000X} % Add \orcidA{} behind the author's name
%\newcommand{\orcidauthorB}{0000-0000-0000-000X} % Add \orcidB{} behind the author's name

% Authors, for the paper (add full first names)
\Author{Firstname Lastname $^{1,\dagger,\ddagger}$\orcidA{}, Firstname Lastname $^{2,\ddagger}$ and Firstname Lastname $^{2,}$*}

%\longauthorlist{yes}

% MDPI internal command: Authors, for metadata in PDF
\AuthorNames{Firstname Lastname, Firstname Lastname and Firstname Lastname}

% MDPI internal command: Authors, for citation in the left column
\AuthorCitation{Lastname, F.; Lastname, F.; Lastname, F.}
% If this is a Chicago style journal: Lastname, Firstname, Firstname Lastname, and Firstname Lastname.

% Affiliations / Addresses (Add [1] after \address if there is only one affiliation.)
\address{%
$^{1}$ \quad Affiliation 1; e-mail@e-mail.com\\
$^{2}$ \quad Affiliation 2; e-mail@e-mail.com}

% Contact information of the corresponding author
\corres{Correspondence: e-mail@e-mail.com; Tel.: (optional; include country code; if there are multiple corresponding authors, add author initials) +xx-xxxx-xxx-xxxx (F.L.)}

% Current address and/or shared authorship
\firstnote{Current address: Affiliation 3.} 
\secondnote{These authors contributed equally to this work.}
% The commands \thirdnote{} till \eighthnote{} are available for further notes

%\simplesumm{} % Simple summary

%\conference{} % An extended version of a conference paper

% Abstract (Do not insert blank lines, i.e. \\) 
\abstract{To be completed.}

% Keywords
\keyword{Gauge equation, spectral sequence, KV-cohomology, hessian manifold, statistical manifold.} 

% The fields PACS, MSC, and JEL may be left empty or commented out if not applicable
%\PACS{J0101}
%\MSC{}
%\JEL{}

%%%%%%%%%%%%%%%%%%%%%%%%%%%%%%%%%%%%%%%%%%
% Only for the journal Diversity
%\LSID{\url{http://}}

%%%%%%%%%%%%%%%%%%%%%%%%%%%%%%%%%%%%%%%%%%
% Only for the journal Applied Sciences
%\featuredapplication{Authors are encouraged to provide a concise description of the specific application or a potential application of the work. This section is not mandatory.}
%%%%%%%%%%%%%%%%%%%%%%%%%%%%%%%%%%%%%%%%%%

%%%%%%%%%%%%%%%%%%%%%%%%%%%%%%%%%%%%%%%%%%
% Only for the journal Data
%\dataset{DOI number or link to the deposited data set if the data set is published separately. If the data set shall be published as a supplement to this paper, this field will be filled by the journal editors. In this case, please submit the data set as a supplement.}
%\datasetlicense{License under which the data set is made available (CC0, CC-BY, CC-BY-SA, CC-BY-NC, etc.)}

%%%%%%%%%%%%%%%%%%%%%%%%%%%%%%%%%%%%%%%%%%
% Only for the journal Toxins
%\keycontribution{The breakthroughs or highlights of the manuscript. Authors can write one or two sentences to describe the most important part of the paper.}

%%%%%%%%%%%%%%%%%%%%%%%%%%%%%%%%%%%%%%%%%%
% Only for the journal Encyclopedia
%\encyclopediadef{For entry manuscripts only: please provide a brief overview of the entry title instead of an abstract.}

%%%%%%%%%%%%%%%%%%%%%%%%%%%%%%%%%%%%%%%%%%
% Only for the journal Advances in Respiratory Medicine
%\addhighlights{yes}
%\renewcommand{\addhighlights}{%

%\noindent This is an obligatory section in “Advances in Respiratory Medicine”, whose goal is to increase the discoverability and readability of the article via search engines and other scholars. Highlights should not be a copy of the abstract, but a simple text allowing the reader to quickly and simplified find out what the article is about and what can be cited from it. Each of these parts should be devoted up to 2~bullet points.\vspace{3pt}\\
%\textbf{What are the main findings?}
% \begin{itemize}[labelsep=2.5mm,topsep=-3pt]
% \item First bullet.
% \item Second bullet.
% \end{itemize}\vspace{3pt}
%\textbf{What is the implication of the main finding?}
% \begin{itemize}[labelsep=2.5mm,topsep=-3pt]
% \item First bullet.
% \item Second bullet.
% \end{itemize}
%}

%%%%%%%%%%%%%%%%%%%%%%%%%%%%%%%%%%%%%%%%%%
\begin{document}

%%%%%%%%%%%%%%%%%%%%%%%%%%%%%%%%%%%%%%%%%%


\section{Introduction}
\subsection{Notations and writing conventions}
All manifolds are assumed to be smooth.
Thorough this document, the next writing conventions are applied:
$M$ is a smooth manifold. For a vector bundle  \begin{tikzcd}
    E \arrow[r,"\pi"] & M
\end{tikzcd}, the notation $\Gamma(U;E)$ with $U \subset M$ an open subset
of the manifold $M$ stands for the $C^\infty(M)$-module of smooth sections over
$U$. The functor $U \mapsto \Gamma(U;E)$ defines a sheaf denoted by $\Gamma_E$. Finally,
$\Gamma(E)$ is a shorthand notation for $\Gamma(M;E).$ Lowercase letters are used for sections,
uppercase ones for tangent vectors.

\section{The gauge equation}
Let \begin{tikzcd}
    E \ar[r,"\pi"] & M 
\end{tikzcd} be a vector bundle. An affine connection $\nabla$ is a $\R$-linear mapping \citep{husemoller2013fibre}:
\begin{equation}
    \label{eq:affine_connection}
   \nabla \colon \Gamma(E) \to \Gamma\left( T^\star M \otimes E \right)
\end{equation}
such that for any $f \in C^\infty(M)$, $\nabla_X fs = df \otimes s + f \nabla_X s.$
Let \begin{tikzcd}
    E^\star \ar[r,"\pi^\star"] & M
\end{tikzcd}
be the bundle obtained by dualizing $E$ fiberwise.  A section $\theta \in \Gamma \left( E^* \otimes E \right)$, that is a $(1,1)$-tensor, defines two bundle morphisms:
 \begin{equation}
    \begin{tikzcd}
        E \ar[r,"\theta"] \ar[rd,"\pi"] & E \ar[d,"\pi"]\\
        & M
    \end{tikzcd} \hspace{40pt}\begin{tikzcd}
        E^\star \ar[r,"\theta^t"] \ar[rd,"\pi^\star"] & E^\star \ar[d,"\pi^\star"]\\
        & M
    \end{tikzcd}
 \end{equation}
 where $\theta^t$ is such that for any $p \in M$, $X \in T_p M, \xi \in T_p^\star M$:
 \begin{equation}
    \label{eq:transpose_theta}
    \left(\theta_p^t \xi  \right)\left( X \right) = \xi \left( \theta_p X \right)
 \end{equation} 
 \begin{defn}
    \label{def:gauge_equation}
    Let $\left( \nabla_1, \nabla_2 \right)$ be a couple of affine connections. A $(1,1)$-tensor $\theta$ is said to be a solution of the gauge equation if for any $s \in \Gamma(E)$:
    \begin{equation}
        \label{eq:gauge_equation}
        \nabla_1 \theta s = \theta \nabla_2 s
    \end{equation}
    or equivalently if the next diagram commutes:
    \begin{equation}
        \label{eq_gauge_diagram}
         \begin{tikzcd}
     \Gamma\left( E \right) \ar[r, "\nabla_2"] \ar[d,"\theta"] & \Gamma\left( T^\star M \otimes E\right) \ar[d,"Id \otimes \theta"] \\
      \Gamma\left(E \right) \ar[r,"\nabla_1"] & \Gamma\left( T^\star M \otimes E\right)
     \end{tikzcd}
    \end{equation}
 \end{defn}
 
 Definition \ref{def:gauge_equation} can be made local, giving rise
 to diagrams:
 \begin{equation}
     \label{eq:eq_local_gauge_diagram}
     \begin{tikzcd}
     \Gamma\left(U; E \right) \ar[r, "\nabla_2"] \ar[d,"\theta_U"] & \Gamma\left( U; T^\star M \otimes E\right) \ar[d,"Id \otimes \theta_U"] \\
      \Gamma\left(U; E \right) \ar[r,"\nabla_1"] & \Gamma\left( U; T^\star M \otimes E\right)
     \end{tikzcd}
 \end{equation}
 with $U$ an open subset of $M$ and $\theta_U \in \Gamma \left(
 U; E^\star \otimes E 
 \right).$
 \begin{defn}
 \label{def:dual_connection}
 Let be $\nabla$ be an affine connection. Its dual is the affine connection:
 \begin{equation}
     \label{eq:dual_connection}
     \nabla^\star \colon  \Gamma(E^\star) \to \Gamma\left( T^\star M \otimes E^\star \right)
 \end{equation}
 defined by the relation:
 \begin{equation}
     \label{eq:nabla_duality_relation}
     \left( \nabla^\star \xi \right)\left(s\right) = 
     d(\xi(s)) - \xi\left(\nabla s\right)
 \end{equation}
 \end{defn}
 \begin{prop}
 \label{prop:dual_gauge}
 If $\theta$ is a solution of the gauge equation with connections $\left(\nabla_1, \nabla_2\right)$, then 
 $\theta^\star$ is a solution of the gauge equation with connections $\left(\nabla_2^\star, \nabla_1^\star\right)$
 \end{prop}
 \begin{proof}
 For $s \in \Gamma(E), \, \xi \in \Gamma(E^\star)$:
 \begin{align}
     \left(\nabla_2^\star (\theta^\star \xi)\right)\left(
     s\right) & = \left(\theta^\star \xi\right)(s) - 
     \left(\theta^\star \xi\right)\nabla_2 s 
     = \xi \left( \theta s \right) - \xi \left( \theta \nabla_2 s \right) \\
     & = \xi \left( \theta s \right) - \xi \left( \nabla_1 
     \theta s \right) 
      = \left(\theta^\star \nabla_1^\star \xi \right)(s)
 \end{align}
 \end{proof}
 Given a couple of connections $\left(\nabla_1 \nabla_2\right)$, the difference $D_{1,2} = \nabla_1 - \nabla_2$ is a section of $\Gamma\left( 
 TM^\star \otimes TM^\star \otimes E\right).$ Using it, the gauge equation \ref{def:gauge_equation} can rewritten as a commutation relation:
 \begin{equation}
     \label{eq:gauge_commutation}
     \nabla_2 \theta - \theta \nabla_2 + D_{1,2} \theta = 0
 \end{equation}
 Let $\tilde{E}$ be the bundle $E \oplus E^\star$. The bilinear form:
 \begin{equation}
     \label{eq:bilinear_etilde}
     B \colon \left(
     X+\alpha,Y+\beta
     \right) \in \tilde{E}^2 \to \beta(X)+\alpha(Y)
 \end{equation}
 is non-degenerate, namely:
 \begin{equation}
 \label{eq:non_degenerate_b}
 \forall Y+\beta \in \tilde{E} B\left( X + \alpha , Y + \beta \right) = 0 \Rightarrow X+\alpha = 0.
 \end{equation}
 \begin{prop}
 \label{prop:b_parallel}
 A $(1,1)$-tensor $\theta$ on $E$ satisfies the gauge equation for a couple of connections $\left( \nabla_1, \nabla_2\right)$ if and only the bilinear form:
 \begin{equation}
     \label{eq:bilinear_theta}
     B_\theta \colon \left(
     X+\alpha,Y+\beta
     \right) \to B\left(X + \alpha, \theta Y + \theta^\star 
     Y \right)
 \end{equation}
 is parallel with respect to the connection $\tilde{\nabla} = \nabla_2 \oplus \nabla_1^\star.$
 \end{prop}
 \begin{proof}
 By definition:
 \begin{equation}
  B\left(X + \alpha, \theta Y + \theta^\star 
     Y \right) = \alpha\left(
     \theta Y \right) + \theta^\star \beta \left( X
     \right)
 \end{equation}
 Taking the differential yields:
 \begin{align}
  d\left(\alpha( \theta Y) \right) & =
 \left(\nabla_1^\star\alpha \right)(\theta Y) +
 \alpha\left(
 \nabla_1 \theta Y
 \right) = \left(\theta^\star `\nabla_1\star \alpha \right) (Y) + 
 \alpha\left(
  \theta \nabla_2 Y
 \right) \\
 & = \left(\nabla_2^\star \theta^\star \alpha \right)(Y)  + 
 \alpha\left(
  \nabla_1 \theta Y
 \right)
 \end{align}
 and symmetrically:
 \begin{align}
   d\left((\theta^\star \beta)( X) \right) & = d\left(\beta)( \theta X) \right) \\
    & = \left(\nabla_2^\star \theta^\star \beta \right)(X)  + 
 \beta\left(
  \nabla_1 \theta X
 \right)
 \end{align}
 Now:
 \begin{align}
 &\tilde{B_\theta}\left(X+\alpha, Y + \beta\right)= \\
 &  
 d{B_\theta}\left(X+\alpha, Y + \beta\right) 
 -B_\theta\left(\tilde{\nabla}(X+\alpha),Y+\beta\right)
 -B_\theta\left(X+\alpha,\tilde{\nabla}(Y+\beta)\right) = \\
 & =  d{B_\theta}\left(X+\alpha, Y + \beta\right) 
 -\beta\left( \nabla_1 \theta X \right) +
 \left(\nabla_2^\star \theta^\star \alpha\right)(Y) 
  -\alpha\left( \nabla_1 \theta X \right) +
 \left(\nabla_2^\star \theta^\star\right) \beta(X) = 0
 \end{align}
 Conversely, if $B_\theta$ is $\tilde{\nabla}$-parallel, then, for any couple $(X+\alpha,Y+\beta)$:
 \begin{equation}
    \alpha \left( \left(\nabla_1 \theta - \theta \nabla_2\right)Y\right) + 
    \beta \left(\left(\nabla_1 \theta - \theta \nabla_2 \right)  X \right) = 0
 \end{equation}
 Taking, for example, $\alpha = 0$, $\beta$ arbitrary, it comes:
 \begin{equation}
     \left(\nabla_1 \theta - \theta \nabla_2 \right)  X = 0
 \end{equation}
 proving that the couple $\left( \theta_1, \theta_2 \right)$ satisfies the gauge equation.
 \end{proof}
 An immediate corollary is:
 \begin{cor}
 \label{cor:invariance_kernel}
 The kernel of $\tilde{\theta} = \theta \oplus \theta^*$ is 
 $\tilde{\nabla}$-invariant, hence the kernel of $\theta$ (resp. $\theta^\star$) is $\nabla_2$ (resp. $\nabla_1^\star$) invariant.
 \end{cor}
 \begin{proof}
 The kernel of $B$ is $\{0\}$, so if:
 \begin{equation}
     \forall Y + \beta \in \tilde{E}, \, B_\theta \left( X+\alpha, Y+\beta\right) = B(\theta X + \theta^\star \alpha, Y + \beta) = 0
 \end{equation}
 Then $\theta X + \theta^\star \alpha = 0$ and $X + \alpha \in \ker \tilde{\theta}.$ Given a basis of $\ker \tilde{\theta}$ at a point $p \in M$, parallel transporting it by $\tilde{\nabla}$ yields another basis of $\ker \tilde{\theta}$ at an arbitrary point $q\in M$, hence the claim.
 \end{proof}
 \begin{rem}
Corollary \ref{cor:invariance_kernel} implies by parallel transport that the dimension of the kernel of $\theta$ (resp. $\theta^\star$) is a constant, hence the rank of $\theta$ (resp. $\theta^\star$) is also a constant. 
 \end{rem}


 
\section{KV cohomology}
\label{sec:kv}
The cochain complex of Koszul-Vinbeg algebras may be introduced by the following three ways:\citep{boyom2016,Boyom2020TheLF}\\
Way1: From the point of view of the tensorial calculus: The brut formula,\\
Way2: From the pointof view of the theory of categories: Simplicial objects.\\
Way3: The point of view of the Anomalies, viz the calculation  rules.\\
In this work, we take into account forthcoming our applications interests in relationships between the information Geometry and the Differential topology.\\
\subsection{Koszul-Vinberg algebras.}
We go to recall useful basic definitions.
\begin{defn} A real Koszul-Vinberg algebra is a real vector space A endowed with a product 
$$A \times A \ni (a,b)\rightarrow ab \in A$$
subject the following identity,
$$(a,b,c) = (b,a,c)$$ 
where
$$(a,b,c) = (ab)c - a(bc).$$
\end{defn}
Examples. \\
(a) Associative algebras are Koszul-Vinberg algebras.\\
(b) The vector of vector fields on a smooth manifold M endowed with symmetric flat Koszul connection $\nabla$.\\
\subsection{KV modules of Koszul-Vinberg algebras}
\begin{defn} A real left module of a real Koszul-Vingerg algebra $A$ is a real vector space $V$ endowed with a bilinear mapping
$$A \times V \ni (a,v) \rightarrow a.v \in V$$
which satisfies the following identity,
$$(a,a',v) = (a',a,v),$$
where
$$(a, a',v) = (aa').v - a.(a'.v).$$
\end{defn}
In this paper we will be dealing with Koszul-Vinberg algebra of vector fields $\mathcal{X}(M)$ on a differentiable manifold $M$ endowed with a Koszul connection $\nabla$ whose both the curvature tensor $R^\nabla$ and the torsion tensor $T^\nabla$ vanish identically.\\
We put 
$$A : = (\mathcal{X}(M), \nabla).$$
The product on $A$ is defined as it follows,
$$a.a' =\nabla_aa'.$$
Of course $\mathcal{X}(M)$ is a left Koszul-Vinberg module of $A$\\
The space of smooth functions $C^\infty(M)$ is a left Koszul-Vinberg module of $A$ under the left action
$$A \times C^\infty(M) \ni (a,f) \rightarrow df(a) \in C^\infty(M)$$
\subsection{Vectorial cochain complexes }
Given a Koszul-Vinberg algebra A, to any left module module V are associated the following two cochain complexes of A with coefficinets in V. One is denoted by $C_{KV}(A,V)$ and is named KV complex. The other is denoted $C_\tau(A,V)$ and is named total KV complex. We go to remind the definition of these complexes and point some domains of their efficiency.\\
\subsubsection{The complex $C_{KV}(A,V)$}
We set 
$$J(V) =\left\{v \in V\quad s.t.\quad (a,a',v) = 0 \quad 
\forall (a,a')\subset A\right\}.$$
Given 
$$\xi = a_1\otimes..\otimes a_{q+1} \in A^{\otimes q+1}$$
we put
$$\partial_i\xi = ..\otimes \hat{a}_i \otimes..$$
and 
$$a.\xi = \Sigma_i ..\otimes a.a_i\otimes..$$

The vector space $C_{KV}(A,V)$ is $\mathbb{Z}$-graded by the homogeneous subspaces $C^q_{KV}$ which are defined as it follows:\\

$C^q_{KV} = 0 \quad if \quad q < 0, $
$C^0_{KV} = J(V)$,\\
$C^q_{KV} = Hom(A^{\otimes q},V),\quad q > 0.$\\
The operator 
$$C^q_{KV}\ni f \rightarrow \delta.f C^{q+1}_{KV}$$
is defined as it follows,
$$\delta.v(a) = a.v \quad (a,v) \in A \times J(V),$$
Let $$f \in C^q_{KV}$$
and 
$$\xi = a_1\otimes.. \otimes a_{q+1}$$
then
$$\delta.f (\xi) = \Sigma^q_1 (-1)^{i-1}[a_i.f(\partial_i\xi) - f(X_i.\partial_i\xi)].$$
The q-th cohomology space is denoted by
$$H^q_{KV}(A,V) = \frac{ker(\delta: C^q_{KV}\rightarrow C^{q+1}_{KV})}{\delta (C^{q-1}_{KV})}.$$
This cohomology is the solution of the conjecture of Gerstenhaber for the deformations of Hyperbolic structure in the sense of Koszul. The conjecture of Gerstenhaber claim that \textit{Every restrict theory of deformation generates it proper theory of cohomology}, \cite{Gerstenhaber1964}.\\
This cohomology also manages equivalence relation between extensions of Koszul-Vinberg algebras by Koszul-Vinberg modules. More explictly let $Ext(A,V)$ the set of equivalence classes of extensions of A by V, then one can write 
$$Ext(A,V) = H^2_{KV}(A,V).$$
In the category of modules of associative algebras and as well as in the category of modules of Lie algebras respectively, the 2nd Hochschild space $HH^2(-,-)$ and the 2nd Chevalley-Eilbenberg space $H^2_{CE}(-,-)$ play the similar role. 
\subsubsection{Total KV complex $C_\tau(A,V)$.}
The cochain complex $C_\tau(A,V)$ is $\mathbb{Z}$-graded by the homogeneous subspaces $C^q_\tau(A,V)$ which as it follows,\\
$C^q_\tau = 0$  if $q < 0$, \\
$C^0_\tau = V$, \\
$C^q_\tau = Hom(A^{\otimes q},V)$ if $q > 0.$\\
We keep using the notation as in the subsubsection above. The operation $\delta_\tau$ is defined as it follows:\\
$\delta_\tau v (a) = a.v$ if  $v\in C^0,$\\
If $f \in C^q_\tau,$\\
$\delta_\tau f(\xi) = \Sigma^{q+1}_1 (-1)^{i+1}[a_i.f(\partial_i\xi) - f(a_i.\partial\xi)]$.\\
The q-th cohomology space of the total complex is denoted by
$$H^q_\tau(A,V) = \frac{Ker(\delta_\tau: C^q_\tau \rightarrow C^{q+1}_\tau)}{\delta_\tau(C^{q-1}_\tau)}.$$
\subsection{Scalar Complexes $C_{KV}(A,R)$ and $C_\tau(A,R)$.}
Henceforth:
$$A = (\mathcal{X}(M),\nabla)$$
and:
$$V = C^\infty(M).$$
The scalar cochain complexes $C_{KV}(A,R)$ is $\mathbb{Z}$-graded as it follows,\\
$C^q_{KV}(A,R) = 0$ for $q < 0$,\\
$C^0_{KV}(A,R) = J(C^\infty(M))$:= the space of affine functions,\\
$C^q_{KV}(A,R) = Hom(A^{\otimes q},C^\infty(R))$ if $q > 0$.\\

Regarding the total KV cohomology the vector space $C_\tau(A,R)$ is graded as it follows,\\  
$C^q_\tau(A,R) = 0$ if $q < 0,$\\
$C^0_\tau(A,R)= C^\infty(M),$ if $q > 0$.\\
$C^q_\tau(A,R) = Hom(A^{\otimes q},C^\infty(M))$ if $q > 0$.\\
\subsection{Links with the De Rham complex.}
We consider the real De Rham complex which is graded as it follows,
$$\Omega(M) = \oplus \Omega^q(M,R),$$
Where
$$\Omega^q(A,R) = Hom(\Lambda^qA,C^\infty(M)).$$

The operator 
$$d:\quad \Omega^q(M)\rightarrow \Omega^{q+1}(M)$$
is defined as it follows,
$$d\omega (a_0 \wedge .. \wedge a_{q} = \Sigma^{q}_0 (-1)^i a_i.\omega(.. \wedge \hat{a}_i \wedge..) + \Sigma_{i < j}(-1)^{i+j}\omega ([a_i,a_j] \wedge.. \wedge \hat{a}_i \wedge.. \hat{a}_j \wedge..).$$
The inclusion map 
$$\Omega^q(M) \subset C^q_\tau(A,R)$$
yields the following cochain complex injective morphism
$$(\Omega(M),d) \rightarrow (C_\tau(A,R),\delta_\tau).$$
The quotient complex is denoted by
$$(Q,d) = \frac{(C_\tau(A,R),\delta)}{(\Omega(M),d)},$$
Links of de Rham cohomology with the total KV cohomology emerge from the following short exact sequence of cochain complexes,
$$O \rightarrow (\Omega(M),d)\rightarrow (C_\tau(A,R),\delta\tau)\rightarrow (Q,d)\rightarrow 0.$$ 
The short exact sequence above yields the following long cohomology exact sequence
$$\rightarrow H^q_{dR}(M,R) \rightarrow H^q_\tau(A,R)\rightarrow H^q(Q)\rightarrow H^{q+1}_{dR}(M,R)\rightarrow$$

\subsection{Tensor product of two KV complexes}
In the category of statistical geometry we will be interested in   spectral sequences which arise from particular double complexes.
We consider two Koszul-Vinberg algebras $A$ and $A^\star$. Let W be of left KV module of the both $A$ and $A^\star$. From this situation arise the following four cochain complexes:\\
$(I): C_{KV}(A,W)$,\\
$(II): C_{KV}(A^\star,W)$,\\
$(III): C_\tau(A,W)$,\\
$(IV): C_\tau(A^\star,W).$\\
Let us consider the vector space $C(W)$ which is by-graded by the vector space
$$ C^{q,p} = C^q_\tau(A,W) \otimes C^p_\tau(A^\star,W).$$
We set
$$C^m = \Sigma_{q+p = m} C^{q,p}.$$
Given $$\alpha \otimes \beta \in C^{q,p}$$
we put
$$\delta(\alpha\otimes\beta) = [ \delta_\tau \alpha\otimes \beta + (-1)^q\alpha \otimes \delta_\tau \beta] \in C^{q+1,p} \oplus C^{q,p+1}$$
It is obvious that 
$$\delta \circ \delta = 0.$$
The cohomology space $H^{q,p}$ is defined as it follows,
$$ H^{q,p}  = \frac{ker(\delta: C^{q,p}\rightarrow C^{q+1,p} \oplus C^{q,p+1})}{im(\delta)\cap C^{q,p}}.$$
In Hessian statistical manifolds solutions of gauge equations give rise to statistical 2-webs. These 2-webs are canonically associated with  tensor products of cochain complexes the cohomology of which can be calculated with spectral sequences.
Situations as in (I), (II), (III), (IV) arise in any Hessian manifold $(M,g,\nabla)$.
\section{Statistical structures}
Henceforth the framework is the category $\gaugecatu$ restricted to objects $(M,g,\nabla)$ where:
\begin{enumerate}[label=(\roman*)]
    \item  $g$ is a Riemannian metric on $TM.$
    \item  $(\nabla,\nabla^+)$ are both torsion free Koszul connections on $TM$.
\end{enumerate}
and morphisms Version March 20, 2024 submitted toof the form $(\text{Id}, \theta).$

We keep the notation in Section 2 and in Section 3.

We are concerned with the gauge equation of the couple $(\nabla,\nabla^\star)$. Let $\theta$ be a solution of that gauge equation, then we define another pair of solutions $(\Theta,\Theta^\star$ by the following identities:
$$2g(\Theta(X),Y) = g(\theta(X), Y) + g(X, \theta(Y)),$$
$$2g(\Theta^\star(X),Y) = g(\theta(X), Y) - g(X, \theta(Y)).$$

All of the following four distributions are regular and are 
in involution,\\
$\left\{Ker(\Theta),Im(\Theta), Ker(\Theta^\star), Im(\Theta^\star)\right\}.$ Furthermore one has the following 2-webs,

$$TM = K \oplus I,$$ 
$$TM = K^\star \oplus I^\star$$ 
where
$$K = ker(\Theta),$$
$$I = im(\Theta),$$
$$K^\star = ker(\Theta^\star),$$
$$I^\star = im(\Theta^\star).$$
Other remarkable properties of these distributions are the following identities:
$$\nabla_XK = K,$$
$$\nabla_XK^\star = K^\star,$$
$$\nabla^\star_XI = I,$$
$$\nabla^\star_XI^\star = I^\star.$$

\textit{FACT.1: If either $(M, g,\nabla)$ or $(M, g, \nabla^\star)$ are Hessian structure then $K$, $K^\star$, $I$ and $I^\star$ are Hessian foliations. Thus any of the pairs $(K,I)$ and $(K^\star,I^\star)$ gives rise to double cochain complex.} 

\textit{FACT.2: It would be remembered that a foliated manifold carries in addition to its total de Rham complex two other remarkable complexes which are the complex of foliated forms and the complex of basic forms.}

\textit{FACT.3: Regrading the picture $\theta\rightarrow (\Theta, \Theta^\star)$ any of the three distributions $ker(\theta)$, $ker(\Theta)$ and $ker(\Theta^\star)$ has a constant rank, nevertheless these three rank may be pairwise different each from other.}

\subsection{Tensor products}
For every non negative integer q the dual vector spaces 
$\Gamma(K^{\otimes q})$ and of $\Gamma(\Lambda^qK)$ are denoted by  $C^q(K)$ and b $\Omega^q_K(M)$ respectively. Let us put
$$C(K) = \oplus_qC^q (K),$$
$$\Omega_K(M) = \oplus_q \Omega^q_K(M).$$

It makes sense to restrict the de Rham operation to $\Omega_K(M)$ in order to define the following cochain complex,
$$\rightarrow \Omega^{q-1}_K(M) \rightarrow \Omega^q_K(M) \rightarrow \Omega^{q+1}_K(M) \rightarrow  $$
In any Hessian manifold we involve \textit{FACT.1} and we use the operators $\delta_{KV}$ and $\delta_\tau$ to write down the KV cochain complexes $[C(K),\delta_\tau]$ and $[C(K),\delta_\tau]$,
$$[C(K),\delta_{KV}]:\quad 0\rightarrow J(C^\infty(M)) \rightarrow ... \rightarrow C^{q-1}(K)\rightarrow C^q(K)\rightarrow C^{q+1}(K)\rightarrow $$
$$[C(K),\delta_\tau]:\quad 0\rightarrow C^\infty(M)\rightarrow..\rightarrow C^{q-1}(K)\rightarrow C^q(K)\rightarrow C^{q+1}(K)\rightarrow $$



The cochain complex above is nothing else than
$$\rightarrow C^{q-1}_{KV}(A_K,R)\rightarrow  C^q_{KV}(A_K,R)\rightarrow C^{q+1}_{KV}(A_K,R)\rightarrow, $$
where
$$A_K = \Gamma(K = ker(\Theta)).$$
Similar complexes are attached to three other distributions $I$, $K^\star$ and $I^\star$.
\subsection{Double complexes in a statistical manifold}
In a statistical manifold $(M,g,\nabla,\nabla^\star)$ let $\theta$ be a solution of the jauge equation of $(\nabla^\star,\nabla)$. The situation just described to the following two 2-webs
$(K,I$ and $(K^\star,I^\star)$.\\
We are concerned with the two de Rham double complexes:  
$$\Omega(K,I) = \oplus_{q,p}\Omega_K(M)\otimes \Omega^p_I(M),$$
$$\Omega(K^\star,I^\star) = \oplus_{q,p}\Omega^q_{K^\star}(M)\otimes\Omega^p_{I^\star}(M)$$
\subsection{Double complexes in a Hessian manifold}
In a Hessian manifold $(M,g,\nabla,\nabla^\star)$ we are concerned with de Rham double complexes which derive from the following 2-webs
$$\Omega(K,I)$$
$$\Omega(K^\star,I^\star)$$
To investigate properies of the statistical structure  $(g,\nabla,\nabla^\star)$ we involve the two Koszul-Vinberg algebras:
$$A = \Gamma(TM), \nabla),$$
$$A^\star = \Gamma(TM),\nabla^\star).$$
Thus we face four complexes:
$$C_{KV}(A,R),$$
$$C_{KV}(A^\star,R),$$
$$C_\tau(A,R),$$
$$C_\tau(A^\star,R).$$
We are interested two double KV complexes
$$C_{KV}(A,A^\star) = \oplus_{q,p}C^q_{KV}(A,R)\otimes C^p_{KV}(A^\star,R)$$
$$C_\tau(A,A^\star) = \oplus C^q_\tau(A,R)\otimes C^p_\tau(A^\star,R).$$
These double complexes give rise to the total complexes $$(C_{KV}(M),d_{KV})$$
and $$(C_\tau(M), d_\tau).$$
Here
$$C_\tau(M) = \oplus_n C^n_\tau(M)$$
where $$C^n_\tau(M) = \oplus_{[q+p = n]}C^q\tau(A,R)\otimes C^p_\tau(A^\star,R).$$
The following operator
$$d_\tau:\quad C^n_\tau(M)\rightarrow C^{n+1}_\tau(M)$$
is defined as it follows, given 
$$u\otimes v \in C^q_\tau(A,R)\otimes C^p_\tau(A^\star,R)$$

$$d_\tau(u\otimes v) = \delta_\tau(u)\otimes v + (-1)^q u\otimes \delta_\tau(v).$$
Mutatis mutandis $(C_{KV},d_{KV})$ is defined similarly.\\
Let G be the group of symmetries of $(M,g,\nabla,\nabla^\star)$; G is the following finite dimensional Lie group 
$$G = Isom(M,g)\cap Aff(M,\nabla).$$

Of course the cohomology spaces of the complexes which are introduced above are geometric invariant of G.\\

\textit{}A forthcoming section will be devoted to calculate the cohomology of these complexes. Before dealing with these calculations we go introduction other useful materials which will be involved. These materials derive from persistent simplicial homologies which are related with the gauge equation as in the subsection just below.\\
\subsection{Gauge equation and homology persistence.}
We fix a statistical structure $(M,g,\nabla,\nabla^\star).$ Let $\theta$ be a solution of the gauge equation of $(\nabla^\star,\nabla)$. According to the notation used in the precedent sections $\theta$ gives rise to two 2-webs
$$(K,I)$$ and $$(K^\star,I^\star).$$
The foliation defined by $I^\star$ is denoted by $\mathcal{F}_\theta$.\\
Let $r^\star(\theta)$ be the rank of the distribution $I^\star$. We put
$$r^\star(M) = \max_\theta\left\{r^\star(\theta)\right\}.$$
STEP.1:\\
Now we choose a $\theta_1$ such that
$$r^\star(\theta_1) = r^\star(M).$$
Henceforth we fix $x \in M$.\\
Let $F_1(x)$ be the leaf of $\mathcal{F}_{\theta_1}$ which contains $x$.\\
So we introduce the $r^\star(M)$-dimensional statistical submanifold $$(F_1(x),g,\nabla^\star) \subset (M,g,\nabla^\star).$$ 
STEP.2:\\
We use the gauge equation of $F_x(g,\nabla^\star)$ to define $r^\star(F_1(x)$. Then we impliment the manipulation as in STEP.1 to introduce the statistical submanifold
$$ (F_2(x),g,\nabla^\star) \subset (F_1(x),g,\nabla^\star).$$
We set 
$$(F_0,g,\nabla^\star) = (M,g,\nabla^\star).$$
Then inductively we construct the following statistical filtration:

\begin{equation} SP(M):(F_q(x),g,\nabla^\star)\subset (F_{q+1}(x),g,\nabla^\star)\subset .. \subset (M,g,\nabla^\star).
\end{equation}


We consider the real singular chain complex of $M$:
\begin{equation}
   Sing(M): \quad \rightarrow C_{q+1}(M)\rightarrow C_q(M)\rightarrow C_{q-1}(M) \rightarrow 
\end{equation}
The topology persistence $(SP(M)$ yields the following homology persistence:
\begin{equation}
  HP(M):\quad \rightarrow Sing(F_{q+1}(x))\rightarrow Sing(F_q(x))\rightarrow Sing(F_{q-1}(x))\rightarrow  
\end{equation}

\section{Spectral sequences}
\label{sec:spectral}
In this section, we briefly recall the definition of spectral sequences of cochain complexes. A good recent reference on the subject is \citep{McCleary_2000}.
\begin{defn}
    \label{def:differential_sheaf}
    A graded differential sheaf $\left(\mathcal{S}, d\right)$ is a graded sheaf $\left(\mathcal{S}^p\right)_{p \in \Z}$ together with a graded morphism $d \colon \mathcal{S}^p \to \mathcal{S}^{p+1}$ satisfying $d^2 = 0.$
    \end{defn}
\begin{defn}
The derived cohomology sheaf is the graded sheaf $H\left(\right)$:
\begin{equation}
    \label{eq:derived_cohomology}
    H^p(\mathcal{S}) = \frac{\ker\{d^p \colon \mathcal{S}^p \to \mathcal{S}^{p+1}\}}
    {\im\{d^{p-1} \colon \mathcal{S}^{p-1} \to \mathcal{S}^p\} }
\end{equation}
\end{defn}
\begin{rem}
The derived cohomology sheaf is the sheafification of the local cohomology presheaf:
\[
U \mapsto H^p \left(\mathcal{S}(U)\right)
\]
\end{rem}
In the sequel, a ring $R$ is fixed.
\begin{defn}
 A bigraded module $E$ over $R$ is a double indexed collection of $R$-modules $E^{p,q}, \, p,q \in \Z.$
\end{defn}
\begin{defn}
Let $E$ be a bigraded module over $R$ and let $r \in \N$. A differential over $E$ of bidegree $(r,1-r)$ is double indexed collections of $R$-morphisms $d \colon E^{p,q} \to E^{p+r,q+1-r}$ such that $d^2 = 0.$
\end{defn}
\begin{defn}
A differential bigraded $R$-module is a couple $(E,d)$ with $E$ a bigraded module and $d$ a differential of bidegree $(r,1-r)$, $r$ a fixed integer.
\end{defn}
\begin{defn} A cohomology spectral sequence is a sequence of bigraded differential modules $(E_r,d_r), r=1,2,\dots$ where $d_r$ has bidegree $(r,1-r)$ and for all $p,q,r$, $E^{p,q}_{r+1} \sim H^{p,q}(E_r,d_r).$
\end{defn}
\begin{rem}
    A spectral sequence can be viewed as a successive approximation process and in most cases, $(E_2,d_2)$ is known and is the starting point of the sequence. Now, looking at stage $n$, that is $(E_n, d_n)$, the defining property of the spectral sequence indicates that if $Z_n = \ker d_n, B_n = \im d_{n-1}$, then, as a bigraded module, $E_{n+1} \sim Z_n / B_n.$ Now, if $\bar{Z}_{n+1} = \ker d_{n+1}, \bar{B}_{n+1}=\im d_n$, there exist modules $Z_{n+1},B_{n+1}$ such that $\bar{Z}_{n+1} = Z_{n+1}/B_{n}$, $\bar{B}_{n+1} = B_{n+1}/B_n$ and, by Noether's isomorphism, $Z_{n+1}/B_{n+1} = \bar{Z}_{n+1}/\bar{B}_{n+1}.$ Furthermore, since $d_{n+1}$ is a differential, $\bar{B_{n+1}} \supset B_n, \, \bar{z_{n+1}} \subset Z_n$, hence $B_n \subset B_{n+1} \subset Z_{n+1} \subset Z_n.$ Proceeding by recurrence, there exist limiting modules:
    \[
    B_\infty = \cup_n B_n, \, Z_\infty = \cap_n Z_n
    \]
    and the purpose of the spectral sequence is to obtain $Z_\infty / B_\infty.$
\end{rem}
\begin{defn}
A spectral sequence is said to converge if there exists, for each couple of integers $(p,q)$ an integer $r(p,q)$ such that all differentials $d_r \colon E^{p,q}_r \to E_{p+r,q+1-r}$ are $0$ for $r \geq r(p,q).$
\end{defn}
\begin{prop}
\label{prop:spectral_limit}
If a spectral sequence converges, then, for any couple of integers $p,q$, the module 
$E_\infty^{p,q}$ is isomorphic to the direct limit of the diagram:
\begin{equation}
\label{eq:spectral_limit}
    \begin{tikzcd}
        E_{r(p,q)}^{p,q} \ar[r] \ar[rd] & E_{r(p,q)+1}^{p,q} \ar[r] \ar[d]& \dots \ar[ld] \\
        & E_\infty^{p,q} & 
    \end{tikzcd}
\end{equation}
\end{prop}
\begin{defn}
    \label{def:exact_couple}
An exact couple is a pair of modules $M,E$ and morphisms $i,j,k$ fitting in the exact diagram:
\begin{equation}
    \label{eq:exact_couple}
    \begin{tikzcd}
        M \ar[rr,"i"] & & M \ar[ld,"j"] \\
        & E \ar[lu,"k"] &
    \end{tikzcd}
\end{equation}
\end{defn}
\begin{prop}
\label{prop:differential_module_couple}
Given an exact couple as in definition \ref{def:exact_couple}, $E$ is differential module with differential $d = j \circ k$
\end{prop}
The next proposition can be found in \citep{McCleary_2000}.
\begin{prop}
\label{prop:derived_couple}
Let $(M,E,i,j,k)$ be an exact couple. The derived couple:
$M_1 = \im (i), E_1 = H(E)$ is exact with morphisms: 
\[
i_1 = i\vert_{M_1}, \, j_1 = j \circ i + d E, \, k_1(e + dE) = k(e) 
\]
\end{prop}
Passing to bigraded modules and iterating the process defines a spectral sequence $(E_r,d_r)$, where $E_r$ is the $r$-th derived module of $E$ and $d_r = j_r \circ k_r.$

Finally, still using \citep{McCleary_2000}, a filtered complex $F^p \mathcal{C} \subset F^{p+1} \mathcal{C} \subset \dots $ defines an exact couple by passing to cohomology. Namely, starting with the short exact sequence:
\begin{equation}
    \label{eq:short_exact}
    \begin{tikzcd}
    0 \ar[r] & F^p \mathcal{C} \ar[r] & F^{p+1}\mathcal{C} \ar[r] & F^{p+1}\mathcal{C}/F^p\mathcal{C} \ar[r] & 0
    \end{tikzcd}
\end{equation}
one obtain a long homology sequence:
\begin{equation}
    \label{eq:long_homology}
    \begin{tikzcd}[column sep=small]
    \dots \ar[r] & H^{p+q}\left(F^{p+1} \mathcal{C}\right) \ar[r,"i"] & H^{p+q}\left(F^{p}\mathcal{C}\right) \ar[r,"j"] & H^{p+q}\left(F^{p+1}\mathcal{C}/F^p\mathcal{C}\right) \ar[r,"k"] & H^{p+q+1} \left(F^{p+1} \mathcal{C}\right) 
    \end{tikzcd}
\end{equation}
Putting:
\[
E^{p,q}=H^{p+q}\left(F^{p+1}\mathcal{C}/F^p\mathcal{C}\right), \, D^{p,q} = H^{p+q}\left(F^{p}\mathcal{C}\right)
\]
one obtain an exact couple, hence a spectral sequence. This construction will be part of the next section,  where our aim will be to point out that from the methods of the information geometry emerge relevant spectral sequences.
\section{Application to statistical manifolds}
\label{sec:statistical}

In an introductory manner the notion a spectral sequence has been reminded in Section 5. As machineries, spectral sequences are powerful tools for homological calculations. 

In a statistical structure $(M,g,\nabla,\nabla^\star)$ we have identified chain complexes and cochain complexes which are attached to solutions of the gauge equation of ($(\nabla^\star,\nabla)$. 

The homology of these complexes may be approximated by some well known spectral sequences. We aim to point out some spectral sequences which are linked with the complexes which have been introduced in the precedent sections. 


\subsection{The spectral sequences of a double complex}
We focus the complexes which emerge from the statistical geometry.

In a Hessian structure $(M,g,\nabla,\nabla^\star)$ we fix a solution $\theta$ of the gauge equation of $(\nabla^\star,\nabla)$.
We focus on the total KV complex
$$C_\tau(M)^n = \oplus_{[j+i = n]}C^j_\tau(A,R)\otimes C^i_\tau(A^\star,R).$$
\subsubsection{The filtration $F^p_A(C_\tau(M))$.}
Before pursuing we define $(d'_\tau,d"_\tau)$ as it follows:given
$$u\otimes v \in C^j_\tau(A,R)\otimes C^i_\tau(A^\star, R),$$
$$d'_\tau(u\otimes v) = \delta_\tau (u) \otimes v,$$
$$d"_\tau(u\otimes v) = (-1)^j u\otimes \delta_\tau(v).$$
$$d"_\tau = 1 \otimes \delta_\tau$$
so that without any confusion one has
$$d_\tau = d'_\tau + d"_\tau.$$
To any couple $(p \leq n)$ of positive integer  we set 
 $$ F^p_{A,n}(C_\tau(M)) = \oplus_{[j\leq p]}C^j_\tau(A,R)\otimes 
C^{n-j}_\tau(A^\star,R), $$

$$ F^p_{A^\star,n}(C_\tau(M)) = \oplus_{[j\leq p]}C^{n-j}_\tau(A,R)\otimes C^j_\tau(A^\star,R).$$
It is easy to verify the following claims,
$$F^p_{A,n}(C_\tau(M)) \subset F^{p+1}_{A,n}(C_\tau(M))$$
$$d"_\tau (F^p_{A,n}(C_\tau(M)) \subset F^p_{A,n+1}(C_\tau(M));$$
$$F^p_{A^\star,n}(C_\tau(M)) \subset F^{p+1}_{A^\star,n}(C_\tau(M)),$$
$$d'_\tau(F^p_{A^\star,n})(C_\tau(M)) \subset F^p_{A^\star, n+1}(C_\tau(M)).$$
Each filtration yields a spectral sequence that we denote by
$$(E_r,d_r(A)),$$
$$E_r(A^\star, d_r).$$
Consider the tensor
$$\Omega_\tau = \Omega(M)\otimes\Omega(M),d_R$$
where $\Omega(M)$ is the de Rham complex of $M$.\\
\textit{Remark,\citep{boyom2016}\\
\textit{The inclusion mapping $$\Omega_\tau(M)\rightarrow C_\tau(M)$$}
 is a complex morphism.}
\subsection{Singular persistence of a compact statistical manifold}
Let $(M,g,\nabla,\nabla^\star)$ be a compact statistical manifolds.
We have involved solutions of gauge equation of $(\nabla^\star,\nabla)$ to obtain an homological persistence on $M$:
$$\subset Sing(F_{p+1}(x))\subset Sing(F_p(x))\subset Sing(F_{p-1}(x)) \subset $$
By canonical construction we deduce a 
short exact sequences 
$$ 0\rightarrow Sing(F^{p+1}(x))\rightarrow Sing(F^p(x)) \rightarrow E^p(x)\rightarrow 0,$$
 with $$E^p(x) = \frac{Sing(F^p(x))}{Sing(F^{p+1}(x))}.$$
This sort exact sequence yields the following long exact sequence of singular homology spaces
$$\rightarrow H_{q+1}(F^{p+1}(x)) \rightarrow H_{q+1}(F^p(x))\rightarrow H_{q+1}(E^p(x))\rightarrow H_q(F^{p+1}(x))\rightarrow$$
One uses the topology persistence to construct an homological exact couple whose spectral sequence converge to the singular homology $H(M)$.\\ 
By theorem of de Rham the approach leads to de Rham algebra of $M$.\\
Indeed by setting
$$M = \oplus_p H(F^p(x)),$$
$$E = \oplus_p H(E^p),$$
the long exact homology sequence above yields the exact couple
$$i: M \rightarrow M,$$
$$j: M \rightarrow E,$$
$$k: E \rightarrow M. $$ 
In section 5 it is explained how to construct inductively the derived exact couples.
Here let us sketch another construction to be applied to the total complex of a Hessian structure $(M,g,\nabla,\nabla^\star)$, namely  
$$C_\tau(M)$$
which is bi-graded by the sub-spaces 
$$C^{q,p}_\tau(M) = C^q_\tau (A,R)\otimes C^p_\tau(A^\star)$$
We remind the filtration
$$F^p_{A,n}(C_\tau(M)) = \oplus_{[j \leq p]}C^j_\tau(A)\otimes C^{n-j}_\tau(A^\star).$$
\textbf{At one side}, according to general process this filtration gives rise to an exact couple which yields a spectral sequence 
$$E(A)= \left\{E^{j,i}_r\right\}$$
\textbf{At another side}, the operators $d'_\tau$ and $d"_\tau$ are defined  as in 6.1.\\
Set
$$ H"^{j,i}(C_\tau(M)) = \frac{ker(d"_\tau: C^{j,i}_\tau(M)\rightarrow C^{j,i+1}_\tau(M))}{d"_\tau(C^{j,i-1}(M))},$$
and 
$$H'^jH"^i(C_\tau(M)) = \frac{ker(d'_\tau: H"^{j,i}(C_\tau(M))\rightarrow H"^{j+1,i}(C_\tau(M))}{d '_\tau(H"^{j-1,i}(C_\tau(M))}.$$

Now we are in position to implement some powerful well known machineries.

\begin{thm} The term $E^{j,i}_2$ of the spectral sequence $E(A)$ is isomorphic to $H^{\prime j} H^{\prime\prime i}(C_\tau(M))\clubsuit$
\end{thm}

\begin{thm} The spectral sequence $E(A)$ converges to the total cohomology of the total complex
$$(C_\tau(M),d_\tau) \clubsuit$$
\end{thm}

For technical details readers are referred to \citep{McCleary_2000}, \citep{Whitehead1960}, \citep{maclane2012homology}.  


\input{conclusion}




%%%%%%%%%%%%%%%%%%%%%%%%%%%%%%%%%%%%%%%%%%
\authorcontributions{The authors have contributed equally to this work.}

\funding{This research received no external funding}



% Only for journal Nursing Reports
%\publicinvolvement{Please describe how the public (patients, consumers, carers) were involved in the research. Consider reporting against the GRIPP2 (Guidance for Reporting Involvement of Patients and the Public) checklist. If the public were not involved in any aspect of the research add: ``No public involvement in any aspect of this research''.}

% Only for journal Nursing Reports
%\guidelinesstandards{Please add a statement indicating which reporting guideline was used when drafting the report. For example, ``This manuscript was drafted against the XXX (the full name of reporting guidelines and citation) for XXX (type of research) research''. A complete list of reporting guidelines can be accessed via the equator network: \url{https://www.equator-network.org/}.}

\acknowledgments{In this section you can acknowledge any support given which is not covered by the author contribution or funding sections. This may include administrative and technical support, or donations in kind (e.g., materials used for experiments).}

\conflictsofinterest{ The authors declare no conflicts of interest.} 

%%%%%%%%%%%%%%%%%%%%%%%%%%%%%%%%%%%%%%%%%%
%% Optional

%% Only for journal Encyclopedia
%\entrylink{The Link to this entry published on the encyclopedia platform.}

% \abbreviations{Abbreviations}{
% The following abbreviations are used in this manuscript:\\

% \noindent 
% \begin{tabular}{@{}ll}
% MDPI & Multidisciplinary Digital Publishing Institute\\
% DOAJ & Directory of open access journals\\
% TLA & Three letter acronym\\
% LD & Linear dichroism
% \end{tabular}
% }

%%%%%%%%%%%%%%%%%%%%%%%%%%%%%%%%%%%%%%%%%%
%% Optional
% \appendixtitles{no} % Leave argument "no" if all appendix headings stay EMPTY (then no dot is printed after "Appendix A"). If the appendix sections contain a heading then change the argument to "yes".
% \appendixstart
% \appendix
% \section[\appendixname~\thesection]{}
% \subsection[\appendixname~\thesubsection]{}
% The appendix is an optional section that can contain details and data supplemental to the main text---for example, explanations of experimental details that would disrupt the flow of the main text but nonetheless remain crucial to understanding and reproducing the research shown; figures of replicates for experiments of which representative data are shown in the main text can be added here if brief, or as Supplementary Data. Mathematical proofs of results not central to the paper can be added as an appendix.

% \begin{table}[H] 
% \caption{This is a table caption.\label{tab5}}
% \newcolumntype{C}{>{\centering\arraybackslash}X}
% \begin{tabularx}{\textwidth}{CCC}
% \toprule
% \textbf{Title 1}	& \textbf{Title 2}	& \textbf{Title 3}\\
% \midrule
% Entry 1		& Data			& Data\\
% Entry 2		& Data			& Data\\
% \bottomrule
% \end{tabularx}
% \end{table}

% \section[\appendixname~\thesection]{}
% All appendix sections must be cited in the main text. In the appendices, Figures, Tables, etc. should be labeled, starting with ``A''---e.g., Figure A1, Figure A2, etc.

%%%%%%%%%%%%%%%%%%%%%%%%%%%%%%%%%%%%%%%%%%
\begin{adjustwidth}{-\extralength}{0cm}
%\printendnotes[custom] % Un-comment to print a list of endnotes

\reftitle{References}

% Please provide either the correct journal abbreviation (e.g. according to the “List of Title Word Abbreviations” http://www.issn.org/services/online-services/access-to-the-ltwa/) or the full name of the journal.
% Citations and References in Supplementary files are permitted provided that they also appear in the reference list here. 

%=====================================
% References, variant A: external bibliography
%=====================================
%\bibliography{your_external_BibTeX_file}

%=====================================
% References, variant B: internal bibliography
%=====================================
\bibliography{main}

% If authors have biography, please use the format below
%\section*{Short Biography of Authors}
%\bio
%{\raisebox{-0.35cm}{\includegraphics[width=3.5cm,height=5.3cm,clip,keepaspectratio]{Definitions/author1.pdf}}}
%{\textbf{Firstname Lastname} Biography of first author}
%
%\bio
%{\raisebox{-0.35cm}{\includegraphics[width=3.5cm,height=5.3cm,clip,keepaspectratio]{Definitions/author2.jpg}}}
%{\textbf{Firstname Lastname} Biography of second author}

% For the MDPI journals use author-date citation, please follow the formatting guidelines on http://www.mdpi.com/authors/references
% To cite two works by the same author: \citeauthor{ref-journal-1a} (\citeyear{ref-journal-1a}, \citeyear{ref-journal-1b}). This produces: Whittaker (1967, 1975)
% To cite two works by the same author with specific pages: \citeauthor{ref-journal-3a} (\citeyear{ref-journal-3a}, p. 328; \citeyear{ref-journal-3b}, p.475). This produces: Wong (1999, p. 328; 2000, p. 475)

%%%%%%%%%%%%%%%%%%%%%%%%%%%%%%%%%%%%%%%%%%
%% for journal Sci
%\reviewreports{\\
%Reviewer 1 comments and authors’ response\\
%Reviewer 2 comments and authors’ response\\
%Reviewer 3 comments and authors’ response
%}
%%%%%%%%%%%%%%%%%%%%%%%%%%%%%%%%%%%%%%%%%%
\PublishersNote{}
\end{adjustwidth}
\end{document}

