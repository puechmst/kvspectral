\section{The gauge equation}
Let \begin{tikzcd}
    E \ar[r,"\pi"] & M 
\end{tikzcd} be a vector bundle. An affine connection $\nabla$ is a $\R$-linear mapping:
\begin{equation}
    \label{eq:affine_connection}
   \nabla \colon \Gamma(E) \to \Gamma\left( T^\star M \otimes E \right)
\end{equation}
such that for any $f \in C^\infty(M)$, any $s \in \Gamma(E)$ and any tangent vector $X$, $\nabla_X fs = X(f) s + f \nabla_X s.$
Let \begin{tikzcd}
    E^\star \ar[r,"\pi^\star"] & M
\end{tikzcd}
be the bundle obtained by dualizing $E$ fiberwise.  A section $\theta \in \Gamma \left( E^* \otimes E \right)$, that is a $(1,1)$-tensor, defines two bundle morphisms:
 \begin{equation}
    \begin{tikzcd}
        E \ar[r,"\theta"] \ar[rd,"\pi"] & E \ar[d,"\pi"]\\
        & M
    \end{tikzcd} \hspace{40pt}\begin{tikzcd}
        E^\star \ar[r,"\theta^t"] \ar[rd,"\pi^\star"] & E^\star \ar[d,"\pi^\star"]\\
        & M
    \end{tikzcd}
 \end{equation}
 where $\theta^t$ is such that for any $p \in M$, $X \in T_p M, \xi \in T_p^\star M$:
 \begin{equation}
    \label{eq:transpose_theta}
    \left(\theta_p^t \xi  \right)\left( X \right) = \xi \left( \theta_p X \right)
 \end{equation} 
 \begin{defn}
    \label{def:gauge_equation}
    Let $\left( \nabla_1, \nabla_2 \right)$ be a couple of affine connections. A $(1,1)$-tensor $\theta$ is said to be a solution  
of the gauge equation.
 \end{defn}
