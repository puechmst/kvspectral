\section{The gauge equation}
\label{sec:gauge_equation}
Let \begin{tikzcd}
    E \ar[r,"\pi"] & M 
\end{tikzcd} be a vector bundle. A Koszul connection $\nabla$ is a $\R$-linear mapping \citep{husemöller2013fibre}:
\begin{equation}
    \label{eq:affine_connection}
   \nabla \colon \Gamma(E) \to \Gamma\left( T^\star M \otimes E \right)
\end{equation}
such that for any $f \in C^\infty(M)$, $\nabla_X fs = df \otimes s + f \nabla_X s.$
Let \begin{tikzcd}
    E^\star \ar[r,"\pi^\star"] & M
\end{tikzcd}
be the bundle obtained by dualizing $E$ fiberwise.  A section $\theta \in \Gamma \left( E^* \otimes E \right)$, that is a $(1,1)$-tensor, defines two bundle morphisms:
 \begin{equation}
    \begin{tikzcd}
        E \ar[r,"\theta"] \ar[rd,"\pi"] & E \ar[d,"\pi"]\\
        & M
    \end{tikzcd} \hspace{40pt}\begin{tikzcd}
        E^\star \ar[r,"\theta^t"] \ar[rd,"\pi^\star"] & E^\star \ar[d,"\pi^\star"]\\
        & M
    \end{tikzcd}
 \end{equation}
 where $\theta^t$ is such that for any $p \in M$, $X \in T_p M, \xi \in T_p^\star M$:
 \begin{equation}
    \label{eq:transpose_theta}
    \left(\theta_p^t \xi  \right)\left( X \right) = \xi \left( \theta_p X \right)
 \end{equation} 
 \begin{defn}
    \label{def:gauge_equation}
    Let $\left( \nabla_1, \nabla_2 \right)$ be a couple of Koszul connections. A $(1,1)$-tensor $\theta$ is said to be a solution of the gauge equation if for any $s \in \Gamma(E)$:
    \begin{equation}
        \label{eq:gauge_equation}
        \nabla_2 \theta s = \theta \nabla_1 s
    \end{equation}
    or equivalently if the next diagram commutes:
    \begin{equation}
        \label{eq_gauge_diagram}
         \begin{tikzcd}
     \Gamma\left( E \right) \ar[r, "\nabla_1"] \ar[d,"\theta"] & \Gamma\left( T^\star M \otimes E\right) \ar[d,"Id \otimes \theta"] \\
      \Gamma\left(E \right) \ar[r,"\nabla_2"] & \Gamma\left( T^\star M \otimes E\right)
     \end{tikzcd}
    \end{equation}
 \end{defn}
 
 Definition \ref{def:gauge_equation} can be made local, giving rise
 to diagrams:
 \begin{equation}
     \label{eq:eq_local_gauge_diagram}
     \begin{tikzcd}
     \Gamma\left(U; E \right) \ar[r, "\nabla_1"] \ar[d,"\theta_U"] & \Gamma\left( U; T^\star M \otimes E\right) \ar[d,"Id \otimes \theta_U"] \\
      \Gamma\left(U; E \right) \ar[r,"\nabla_2"] & \Gamma\left( U; T^\star M \otimes E\right)
     \end{tikzcd}
 \end{equation}
 with $U$ an open subset of $M$ and $\theta_U \in \Gamma \left(
 U; E^\star \otimes E 
 \right).$
 The above definitions can be generalized to arbitrary vector bundles over $M$ giving rise to a category $\gaugecat$ whose objects are couples $(E,\nabla)$, with $E$ a vector bundle on $M$ and $\nabla$ a Koszul connection on $E$ and morphisms are bundle morphisms $\theta \colon E \to F$ such that $(E, \nabla_1) \to (F, \nabla_2)$ if the diagram:
   \begin{equation}
        \label{eq:gauge_morphism}
         \begin{tikzcd}
     \Gamma\left( E \right) \ar[r, "\nabla_1"] \ar[d,"\theta"] & \Gamma\left( T^\star M \otimes E\right) \ar[d,"Id \otimes \theta"] \\
      \Gamma\left(F \right) \ar[r,"\nabla_2"] & \Gamma\left( T^\star M \otimes F\right)
     \end{tikzcd}
    \end{equation}
 commutes.
 \begin{defn}
 \label{def:dual_connection}
 Let be $\nabla$ be an affine connection. Its dual is the affine connection:
 \begin{equation}
     \label{eq:dual_connection}
     \nabla^\star \colon  \Gamma(E^\star) \to \Gamma\left( T^\star M \otimes E^\star \right)
 \end{equation}
 defined by the relation:
 \begin{equation}
     \label{eq:nabla_duality_relation}
     \left( \nabla^\star \xi \right)\left(s\right) = 
     d(\xi(s)) - \xi\left(\nabla s\right)
 \end{equation}
 \end{defn}
 \begin{prop}
 \label{prop:dual_gauge}
 If $\theta$ is a solution of the gauge equation with connections $\left(\nabla_1, \nabla_2\right)$, then 
 $\theta^\star$ is a solution of the gauge equation with connections $\left(\nabla_2^\star, \nabla_1^\star\right)$
 \end{prop}
 \begin{proof}
 For $s \in \Gamma(E), \, \xi \in \Gamma(E^\star)$:
 \begin{align}
     \left(\nabla_2^\star (\theta^\star \xi)\right)\left(
     s\right) & = \left(\theta^\star \xi\right)(s) - 
     \left(\theta^\star \xi\right)\nabla_2 s 
     = \xi \left( \theta s \right) - \xi \left( \theta \nabla_2 s \right) \\
     & = \xi \left( \theta s \right) - \xi \left( \nabla_1 
     \theta s \right) 
      = \left(\theta^\star \nabla_1^\star \xi \right)(s)
 \end{align}
 \end{proof}
 Given a couple of connections $\left(\nabla_1 \nabla_2\right)$, the difference $D_{1,2} = \nabla_1 - \nabla_2$ is a section of $\Gamma\left( 
 TM^\star \otimes TM^\star \otimes E\right).$ Using it, the gauge equation \ref{def:gauge_equation} can rewritten as:
 \begin{equation}
     \label{eq:gauge_commutation}
     \nabla_2 \theta - \theta \nabla_2 + D_{1,2} \theta = 0
 \end{equation}
Considering $\theta$ as a $O$-form with values in $E^\star \otimes E$, equation \ref{eq:gauge_commutation} may be rewritten as:
\begin{equation}
    d^{\nabla_2} \theta + D_{1,2}\theta = 0
\end{equation}
where $d^{\nabla_2}$ is the exterior covariant derivative
associated with the connection $\nabla_2.$ When $\nabla_2$ is flat, $d^{\nabla_2}d^{\nabla_2} = 0$ and thus 
$d^{\nabla_2}(D_{1,2}\theta) = 0.$

We recall that the Gauge group $\mathcal{G}(E)$ is the set of bundle isomorphisms: 
\begin{equation}
    \label{eq:gauge_group}
    \begin{tikzcd}
    E \ar[r,shift left = 1.5ex, "U"] \ar[rd, shift right=0.5ex] & E \ar[l,"U^{-1}"] \ar[d] \\
    & M
    \end{tikzcd}
\end{equation}
Given a connection $\nabla$, $U \nabla U^{-1}$ is also a connection.
\begin{prop}
\label{prop:conjugate_connections}
Let the triple $\left(\nabla_1, \nabla_2,\theta \right)$ be a solution of the gauge equation $\nabla_1 \theta = \theta \nabla_2.$ 
For any couple $(U,V)$ in $\mathcal{G}(E)$, the triple:
\[
\left( U \nabla_1 U^{-1}, U \theta V^{-1}, V \nabla_2 V^{-1} \right)
\]
is a solution of a gauge equation.
\end{prop}
\begin{proof}
Starting with $\nabla_1 \theta = \theta \nabla_2$, it comes:
\begin{equation}
U^{-1} U \nabla_1 U^{-1} U \theta V^{-1} V = U^{-1} U \theta V^{-1} V \nabla_2 V^{-1} V 
\end{equation}
Composing by $U$ to the left and $V^{-1}$ to the right yields the result.
\end{proof}
Proposition \ref{prop:conjugate_connections} indicates that the existence of a solution does not depend on a particular choice of frame-coframe for representing it. Furthermore, locally, it is always possible to assume a $\theta$ of the form:
\begin{equation}
    \label{eq:reduced_theta}
    \theta = \left( \begin{array}{c|c} \text{Id} & 0 \\ \hline
    0 & 0
    \end{array} \right)
\end{equation}since a pair $U,V$ such that $U \theta V^{-1}$ has the reduced form of eq.\ref{eq:reduced_theta} exists by a standard linear algebra argument. Global reduction is, however, not possible as transitions functions will generally not preserve the diagonal structure.
 Let $\tilde{E}$ be the bundle $E \oplus E^\star$. The bilinear form:
 \begin{equation}
     \label{eq:bilinear_etilde}
     B \colon \left(
     X+\alpha,Y+\beta
     \right) \in \tilde{E}^2 \to \beta(X)+\alpha(Y)
 \end{equation}
 is non-degenerate, namely:
 \begin{equation}
 \label{eq:non_degenerate_b}
 \forall Y+\beta \in \tilde{E} B\left( X + \alpha , Y + \beta \right) = 0 \Rightarrow X+\alpha = 0.
 \end{equation}
 \begin{prop}
 \label{prop:b_parallel}
 A $(1,1)$-tensor $\theta$ on $E$ satisfies the gauge equation for a couple of connections $\left( \nabla_1, \nabla_2\right)$ if and only the bilinear form:
 \begin{equation}
     \label{eq:bilinear_theta}
     B_\theta \colon \left(
     X+\alpha,Y+\beta
     \right) \to B\left(X + \alpha, \theta Y + \theta^\star 
     Y \right)
 \end{equation}
 is parallel with respect to the connection $\tilde{\nabla} = \nabla_2 \oplus \nabla_1^\star.$
 \end{prop}
 \begin{proof}
 By definition:
 \begin{equation}
  B\left(X + \alpha, \theta Y + \theta^\star 
     Y \right) = \alpha\left(
     \theta Y \right) + \theta^\star \beta \left( X
     \right)
 \end{equation}
 Taking the differential yields:
 \begin{align}
  d\left(\alpha( \theta Y) \right) & =
 \left(\nabla_1^\star\alpha \right)(\theta Y) +
 \alpha\left(
 \nabla_1 \theta Y
 \right) = \left(\theta^\star `\nabla_1\star \alpha \right) (Y) + 
 \alpha\left(
  \theta \nabla_2 Y
 \right) \\
 & = \left(\nabla_2^\star \theta^\star \alpha \right)(Y)  + 
 \alpha\left(
  \nabla_1 \theta Y
 \right)
 \end{align}
 and symmetrically:
 \begin{align}
   d\left((\theta^\star \beta)( X) \right) & = d\left(\beta)( \theta X) \right) \\
    & = \left(\nabla_2^\star \theta^\star \beta \right)(X)  + 
 \beta\left(
  \nabla_1 \theta X
 \right)
 \end{align}
 Now:
 \begin{align}
 &\tilde{B_\theta}\left(X+\alpha, Y + \beta\right)= \\
 &  
 d{B_\theta}\left(X+\alpha, Y + \beta\right) 
 -B_\theta\left(\tilde{\nabla}(X+\alpha),Y+\beta\right)
 -B_\theta\left(X+\alpha,\tilde{\nabla}(Y+\beta)\right) = \\
 & =  d{B_\theta}\left(X+\alpha, Y + \beta\right) 
 -\beta\left( \nabla_1 \theta X \right) +
 \left(\nabla_2^\star \theta^\star \alpha\right)(Y) 
  -\alpha\left( \nabla_1 \theta X \right) +
 \left(\nabla_2^\star \theta^\star\right) \beta(X) = 0
 \end{align}
 Conversely, if $B_\theta$ is $\tilde{\nabla}$-parallel, then, for any couple $(X+\alpha,Y+\beta)$:
 \begin{equation}
    \alpha \left( \left(\nabla_1 \theta - \theta \nabla_2\right)Y\right) + 
    \beta \left(\left(\nabla_1 \theta - \theta \nabla_2 \right)  X \right) = 0
 \end{equation}
 Taking, for example, $\alpha = 0$, $\beta$ arbitrary, it comes:
 \begin{equation}
     \left(\nabla_1 \theta - \theta \nabla_2 \right)  X = 0
 \end{equation}
 proving that the couple $\left( \theta_1, \theta_2 \right)$ satisfies the gauge equation.
 \end{proof}
 An immediate corollary is:
 \begin{cor}
 \label{cor:invariance_kernel}
 The kernel of $\tilde{\theta} = \theta \oplus \theta^*$ is 
 $\tilde{\nabla}$-invariant, hence the kernel of $\theta$ (resp. $\theta^\star$) is $\nabla_2$ (resp. $\nabla_1^\star$) invariant.
 \end{cor}
 \begin{proof}
 The kernel of $B$ is $\{0\}$, so if:
 \begin{equation}
     \forall Y + \beta \in \tilde{E}, \, B_\theta \left( X+\alpha, Y+\beta\right) = B(\theta X + \theta^\star \alpha, Y + \beta) = 0
 \end{equation}
 Then $\theta X + \theta^\star \alpha = 0$ and $X + \alpha \in \ker \tilde{\theta}.$ Given a basis of $\ker \tilde{\theta}$ at a point $p \in M$, parallel transporting it by $\tilde{\nabla}$ yields another basis of $\ker \tilde{\theta}$ at an arbitrary point $q\in M$, hence the claim.
 \end{proof}
 \begin{rem}
Corollary \ref{cor:invariance_kernel} implies by parallel transport that the dimension of the kernel of $\theta$ (resp. $\theta^\star$) is a constant, hence the rank of $\theta$ (resp. $\theta^\star$) is also a constant. 
\end{rem}
\begin{rem}
The kernel of $\theta^\star$ is the set of differential forms vanishing on the image of $\theta.$ The knowledge of the kernel of $B_\theta$ thus completely characterize $\ker \theta$ and $\im \theta.$ In particular, $\theta$ has constant rank.
\end{rem}
When there exists a Riemannian metric on the manifold $M$, the gauge equation can be specialized to pairs of connections on $TM$ related by duality. 
\begin{defn}
    \label{def:conjugate_connection}
    Let $\nabla$ be an affine connection. Its conjugate with respect to $g$ (often referred as the dual connection) is the connection $\nabla^+$ defined by the relation:
    \begin{equation}
    \label{eq:conjugate_connection}
    \forall Z \in TM, \forall r,s \in \Gamma \left(TM 
    \right), Z\left( g(r,s) \right) = g\left(\nabla_Z r, s\right) + 
    g\left( r, \nabla_Z^+ s\right)
    \end{equation}
\end{defn}
\begin{rem}
The most common notation for the conjugate connection is $\nabla^\star.$ In the present text, we adopt $\nabla^+$ to distinguish from the connection on $E^\star.$
\end{rem}
\begin{defn}
\label{def:endo_conjugate}
Let $\theta$ be a bundle morphism on $TM.$ Its conjugate,
denoted $\theta^+$, is the bundle morphism defined by:
\begin{equation}
    \label{eq:endo_conjugate}
    \forall X,Y \in TM, \, g\left( \theta X, Y\right) = 
    g\left( X, \theta^+ Y\right)
\end{equation}
\end{defn}
\begin{prop}
\label{prop:inverse_unitary}
If $U \colon TM \to TM$ is a unitary bundle isomorphism, that is:
\[
\forall X,Y \in TM, \, g\left( UX, UY \right) = g(X,Y)
\]
Then $U^{-1} = U^+.$
\end{prop}
\begin{prop}
\label{prop:unitary_conjugate}
Let $\nabla$ be a connection and $U$ be a unitary bundle isomorphism, then:
\begin{equation}
\label{eq:unitary_conjugate}
\left(U \nabla U^+\right)^+ = U \nabla^+ U^+
\end{equation}
\end{prop}
\begin{proof}
If $U$ is unitary, so is $U^+.$
Let $Z \in TM, r,s in \Gamma \left( TM \right)$. Then:
\begin{align}
    Z\left(g(r,s) \right) & = Z\left(g(U^+r,U^+s) \right)\\
    &= g\left(\nabla_Z U^+ r,  u^+ s\right) + g\left(U^+ r, \nabla_Z^+s \right) \\
    &= g\left(U \nabla_Z U^+ r , s\right) +
    g\left(r , U \nabla_Z^+ U^+ s\right)
\end{align}
and the claim follows.
\end{proof}
A is proposition \ref{prop:conjugate_connections}:
\begin{prop}
\label{prop:unitary_conjugate_gauge}
If the triple $\left(\nabla,\nabla^+,\theta \right)$ satisfies the gauge equation $\nabla \theta = \theta \nabla^+$, so does  $\left(U \nabla, U^+, U \nabla^+ U^+, U^+ \theta U\right)$ for any unitary isomorphism $U.$
\end{prop}
\begin{rem}
If $\theta$ is normal, that is $\left [ \theta, \theta^+\right ] = 0$, and the triple $\left(\nabla,\nabla^+,\theta \right)$ satisfies the gauge equation $\nabla \theta = \theta \nabla^+$, then, locally, there exists a unitary isomorphism $U$ such that $U^+ \theta U$ is diagonal and $\left(U \nabla, U^+, U \nabla^+ U^+, U^+ \theta U\right)$ satisfies a gauge equation. This is again a well-known fact from linear algebra since $\theta$ is locally diagonalizable in an orthonormal frame. As in the case of eq. \ref{eq:reduced_theta}, this is generally not true globally.
\end{rem}
\begin{prop}
Using the musical isomorphisms 
    $
    \begin{tikzcd}
    TM \ar[r,shift left, "\flat"] & T^\star M \ar[l,shift left, "\sharp"]
    \end{tikzcd}
    ,$
it comes:
\[
\forall X\in TM,\alpha \in T^\star M \, \left(\nabla^+_X \alpha^\sharp \right) = \left(
\nabla_X \alpha\right)^\sharp
\]
\end{prop}
\begin{proof}
For any $X,Y,Z \in TM$:
\[
Z \left( g(X,Y) \right) = g\left( \nabla_Z X, Y \right) +
g\left( X, \nabla^+_Z Y \right)
\]
Passing to forms, for any $Z \in TM$ $\alpha in T^\star M$, $ X \in TM$:
\[
Z \left( g(X, \alpha^\sharp) \right) = g\left( \nabla_Z X, \alpha^\sharp \right) +
g\left( X, \nabla^+_Z \alpha^\sharp \right)
\]
Now:
\[
\begin{split}
Z \left( g(X, \alpha^\sharp) \right) & = Z \left(\alpha(X)\right) = 
\left(\nabla^\star \alpha \right)(X) + \alpha \left( \nabla_Z X\right) \\
&  g\left(X,\left(\nabla^\star \alpha\right)^\sharp\right) +
g \left(\nabla_Z X , \alpha^\sharp \right)
\end{split}
\]
and the claim follows by identification.
\end{proof}
\begin{rem}
    
\end{rem}
\begin{prop}
\label{prop:parallel_g_tensor}
Let the triple $\left(\nabla,\nabla^+,\theta \right)$ satisfies the gauge equation $\nabla \theta = \theta \nabla^+.$ The tensor:
\begin{equation}
    \label{eq:parallel_g_tensor}
    g_\theta \colon (X,Y) \mapsto g\left( \theta X, Y\right)
\end{equation}
is $\nabla$ parallel.
\end{prop}
\begin{proof}
    The tensor $B_\theta$ of proposition \ref{prop:b_parallel} can be written, using the metric, as:
    \begin{align}
        B_\theta\left( X+ \alpha, Y + \beta \right) &= \beta(\theta X) + \alpha(\theta Y) \\
        & = g \left( \theta X, \beta^\sharp \right) + g \left( \alpha^\sharp , \theta Y \right)
    \end{align}
    Since $B_\theta$ is $\tilde{\nabla}$-parallel, the proposition follows.
\end{proof}
\begin{rem}
    Defining a metric $g^\star$ on $T^\star M$ by:
\begin{equation}
    \label{eq:g_star_metric}
    \forall \alpha, \beta \in T^\star M, \, g^\star(\alpha,\beta) = g\left(
    \alpha^\sharp, \beta^\sharp
    \right)
\end{equation}
The proof of proposition \ref{prop:parallel_g_tensor} also shows that the tensor:
\[
g_\theta^\star \colon (\alpha,\beta) \mapsto g^\star\left(\theta^\star \alpha, \beta\right)
\]
is $\nabla^\star$-parallel.
\end{rem}
Proposition  \ref{prop:parallel_g_tensor} has the important consequence that $TM$ can be split in two ways:
\begin{equation}
    \label{eq:split_tm}
    TM = \ker \theta \oplus \im \theta , \, TM = \ker \theta^+ \oplus \im \theta^+.
\end{equation}
It is clear from proposition \ref{prop:parallel_g_tensor} that if $\theta$ is symmetric, that is $\theta = \theta^+$, the tensor:
\begin{equation}
(X,Y) \mapsto \frac{1}{2} g\left( \theta X, Y\right) + g\left( \theta X, Y\right)
\end{equation}
is $\nabla$-parallel.
When $\theta$ is skew symmetric $\theta = -\theta^+$, the same is true for:
\begin{equation}
(X,Y) \mapsto \frac{1}{2} g\left( \theta X, Y\right) - g\left( \theta X, Y\right)
\end{equation}
As in \ref{eq:gauge_morphism}, a category so that morphisms represent gauge equation solutions. The situation is nevertheless a little bit more complicated as the dimension of the vector bundle may not agree. 
We recall the well-known definition:
\begin{defn}
\label{def:metric_bundle}
Let $E \rightarrow M$ be a vector bundle. A pseudo-riemannian metric on $E$ is a smooth bilinear $C^{\infty}(M)$-mapping $g_E \colon \Gamma(E) \times \Gamma(E) \to C^\infty(M)$ such that:
\begin{itemize}
\item $\forall s,s^\prime \in \Gamma(E), \, g_E(s,s^\prime) = g_E(s^\prime, s).$
\item There exists an isomorphisme ${}^\flat \colon \Gamma(E) \to \Gamma(E^\star)$ such that, for any $s,s^\prime \in \Gamma(E)$: 
\[
s^\flat(s^\prime) = g_E(s,s^\prime).
\]
\end{itemize}
A pseudo-riemannian metric is Riemannian if $g_E(s,s) > 0$ for any $s\neq 0$ in $\Gamma(E).$
\end{defn}
\begin{defn}
\label{def:partial_iso}
Let $E,F$ be two vector bundles on $M$ equipped with respective Riemannian metrics $g_E,g_F.$ A partial isometry from $E$ to $F$ is a bundle morphism $U$ such that the following diagram commutes:
\begin{equation}
    \label{eq:partial_iso}
    \begin{tikzcd}
        E \ar[r,"{}^\flat"] \ar[d,"U"] & E^\star \\
        F \ar[r,"{}^\flat"] & F^\star \ar[u,"U^\star"]
    \end{tikzcd}
\end{equation}
\end{defn}
\begin{rem}
    Definition \ref{def:partial_iso} is equivalent to the fact that for any $s,s^\prime \in \Gamma(E)$:
    \[
    g_F\left(Us, Us^\prime\right) = g_E\left(s,s^\prime\right).
    \]
\end{rem}
\begin{defn}
\label{def:general_dual_connections}
Let $U \colon E \to F$ be a partial isometry and $\nabla_2$ be a Koszul connection on $F$. Its dual $\nabla_2^+$ is the connection on $E$ defined by the relation
\begin{equation}
    \label{eq:general_dual_connections}
    U \nabla_2^+ = \nabla_2 U.
\end{equation}
\end{defn}
\begin{defn}
    \label{def:category_gauge_dual}
    The category $\gaugecatu$ has objects $(E,\nabla)$ where $\nabla$ is a Koszul connection on $E$ and morphisms $(U, \theta) \colon (E, \nabla_1) \to (F, \nabla_2)$ where $U \colon E \to F$ is a partial isometry, $\theta \colon E \to F$ is a bundle morphism and $\nabla_1 = \nabla_2^+, \nabla_2 \theta = \theta \nabla_1.$
\end{defn}
