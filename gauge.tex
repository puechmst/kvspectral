\section{The gauge equation}
Let \begin{tikzcd}
    E \ar[r,"\pi"] & M 
\end{tikzcd} be a vector bundle. An affine connection $\nabla$ is a $\R$-linear mapping \citep{husemöller2013fibre}:
\begin{equation}
    \label{eq:affine_connection}
   \nabla \colon \Gamma(E) \to \Gamma\left( T^\star M \otimes E \right)
\end{equation}
such that for any $f \in C^\infty(M)$, $\nabla_X fs = df \otimes s + f \nabla_X s.$
Let \begin{tikzcd}
    E^\star \ar[r,"\pi^\star"] & M
\end{tikzcd}
be the bundle obtained by dualizing $E$ fiberwise.  A section $\theta \in \Gamma \left( E^* \otimes E \right)$, that is a $(1,1)$-tensor, defines two bundle morphisms:
 \begin{equation}
    \begin{tikzcd}
        E \ar[r,"\theta"] \ar[rd,"\pi"] & E \ar[d,"\pi"]\\
        & M
    \end{tikzcd} \hspace{40pt}\begin{tikzcd}
        E^\star \ar[r,"\theta^t"] \ar[rd,"\pi^\star"] & E^\star \ar[d,"\pi^\star"]\\
        & M
    \end{tikzcd}
 \end{equation}
 where $\theta^t$ is such that for any $p \in M$, $X \in T_p M, \xi \in T_p^\star M$:
 \begin{equation}
    \label{eq:transpose_theta}
    \left(\theta_p^t \xi  \right)\left( X \right) = \xi \left( \theta_p X \right)
 \end{equation} 
 \begin{defn}
    \label{def:gauge_equation}
    Let $\left( \nabla_1, \nabla_2 \right)$ be a couple of affine connections. A $(1,1)$-tensor $\theta$ is said to be a solution of the gauge equation if for any $s \in \Gamma(E)$:
    \begin{equation}
        \label{eq:gauge_equation}
        \nabla_1 \theta s = \theta \nabla_2 s
    \end{equation}
    or equivalently if the next diagram commutes:
    \begin{equation}
        \label{eq_gauge_diagram}
         \begin{tikzcd}
     \Gamma\left( E \right) \ar[r, "\nabla_2"] \ar[d,"\theta"] & \Gamma\left( T^\star M \otimes E\right) \ar[d,"Id \otimes \theta"] \\
      \Gamma\left(E \right) \ar[r,"\nabla_1"] & \Gamma\left( T^\star M \otimes E\right)
     \end{tikzcd}
    \end{equation}
 \end{defn}
 
 Definition \ref{def:gauge_equation} can be made local, giving rise
 to diagrams:
 \begin{equation}
     \label{eq:eq_local_gauge_diagram}
     \begin{tikzcd}
     \Gamma\left(U; E \right) \ar[r, "\nabla_2"] \ar[d,"\theta_U"] & \Gamma\left( U; T^\star M \otimes E\right) \ar[d,"Id \otimes \theta_U"] \\
      \Gamma\left(U; E \right) \ar[r,"\nabla_1"] & \Gamma\left( U; T^\star M \otimes E\right)
     \end{tikzcd}
 \end{equation}
 with $U$ an open subset of $M$ and $\theta_U \in \Gamma \left(
 U; E^\star \otimes E 
 \right).$
 \begin{defn}
 \label{def:dual_connection}
 Let be $\nabla$ be an affine connection. Its dual is the affine connection:
 \begin{equation}
     \label{eq:dual_connection}
     \nabla^\star \colon  \Gamma(E^\star) \to \Gamma\left( T^\star M \otimes E^\star \right)
 \end{equation}
 defined by the relation:
 \begin{equation}
     \label{eq:nabla_duality_relation}
     \left( \nabla^\star \xi \right)\left(s\right) = 
     \xi(s) - \xi\left(\nabla s\right)
 \end{equation}
 \end{defn}
 \begin{prop}
 \label{prop:dual_gauge}
 If $\theta$ is a solution of the gauge equation with connections $\left(\nabla_1, \nabla_2\right)$, then 
 $\theta^\star$ is a solution of the gauge equation with connections $\left(\nabla_2^\star, \nabla_1^\star\right)$
 \end{prop}
 \begin{proof}
 For $s \in \Gamma(E), \, \xi \in \Gamma(E^\star)$:
 \begin{align}
     \left(\nabla_2^\star (\theta^\star \xi)\right)\left(
     s\right) & = \left(\theta^\star \xi\right)(s) - 
     \left(\theta^\star \xi\right)\nabla_2 s 
     = \xi \left( \theta s \right) - \xi \left( \theta \nabla_2 s \right) \\
     & = \xi \left( \theta s \right) - \xi \left( \nabla_1 
     \theta s \right) 
      = \left(\theta^\star \nabla_1^\star \xi \right)(s)
 \end{align}
 \end{proof}
 Given a couple of connections $\left(\nabla_1 \nabla_2\right)$, the difference $D_{1,2} = \nabla_1 - \nabla_2$ is a 
 