\section{The gauge equation}
Let \begin{tikzcd}
    E \ar[r,"\pi"] & M 
\end{tikzcd} be a vector bundle. An affine connection $\nabla$ is a $\R$-linear mapping \citep{husemoller2013fibre}:
\begin{equation}
    \label{eq:affine_connection}
   \nabla \colon \Gamma(E) \to \Gamma\left( T^\star M \otimes E \right)
\end{equation}
such that for any $f \in C^\infty(M)$, $\nabla_X fs = df \otimes s + f \nabla_X s.$
Let \begin{tikzcd}
    E^\star \ar[r,"\pi^\star"] & M
\end{tikzcd}
be the bundle obtained by dualizing $E$ fiberwise.  A section $\theta \in \Gamma \left( E^* \otimes E \right)$, that is a $(1,1)$-tensor, defines two bundle morphisms:
 \begin{equation}
    \begin{tikzcd}
        E \ar[r,"\theta"] \ar[rd,"\pi"] & E \ar[d,"\pi"]\\
        & M
    \end{tikzcd} \hspace{40pt}\begin{tikzcd}
        E^\star \ar[r,"\theta^t"] \ar[rd,"\pi^\star"] & E^\star \ar[d,"\pi^\star"]\\
        & M
    \end{tikzcd}
 \end{equation}
 where $\theta^t$ is such that for any $p \in M$, $X \in T_p M, \xi \in T_p^\star M$:
 \begin{equation}
    \label{eq:transpose_theta}
    \left(\theta_p^t \xi  \right)\left( X \right) = \xi \left( \theta_p X \right)
 \end{equation} 
 \begin{defn}
    \label{def:gauge_equation}
    Let $\left( \nabla_1, \nabla_2 \right)$ be a couple of affine connections. A $(1,1)$-tensor $\theta$ is said to be a solution of the gauge equation if for any $s \in \Gamma(E)$:
    \begin{equation}
        \label{eq:gauge_equation}
        \nabla_1 \theta s = \theta \nabla_2 s
    \end{equation}
    or equivalently if the next diagram commutes:
    \begin{equation}
        \label{eq_gauge_diagram}
         \begin{tikzcd}
     \Gamma\left( E \right) \ar[r, "\nabla_2"] \ar[d,"\theta"] & \Gamma\left( T^\star M \otimes E\right) \ar[d,"Id \otimes \theta"] \\
      \Gamma\left(E \right) \ar[r,"\nabla_1"] & \Gamma\left( T^\star M \otimes E\right)
     \end{tikzcd}
    \end{equation}
 \end{defn}
 
 Definition \ref{def:gauge_equation} can be made local, giving rise
 to diagrams:
 \begin{equation}
     \label{eq:eq_local_gauge_diagram}
     \begin{tikzcd}
     \Gamma\left(U; E \right) \ar[r, "\nabla_2"] \ar[d,"\theta_U"] & \Gamma\left( U; T^\star M \otimes E\right) \ar[d,"Id \otimes \theta_U"] \\
      \Gamma\left(U; E \right) \ar[r,"\nabla_1"] & \Gamma\left( U; T^\star M \otimes E\right)
     \end{tikzcd}
 \end{equation}
 with $U$ an open subset of $M$ and $\theta_U \in \Gamma \left(
 U; E^\star \otimes E 
 \right).$
 \begin{defn}
 \label{def:dual_connection}
 Let be $\nabla$ be an affine connection. Its dual is the affine connection:
 \begin{equation}
     \label{eq:dual_connection}
     \nabla^\star \colon  \Gamma(E^\star) \to \Gamma\left( T^\star M \otimes E^\star \right)
 \end{equation}
 defined by the relation:
 \begin{equation}
     \label{eq:nabla_duality_relation}
     \left( \nabla^\star \xi \right)\left(s\right) = 
     d(\xi(s)) - \xi\left(\nabla s\right)
 \end{equation}
 \end{defn}
 \begin{prop}
 \label{prop:dual_gauge}
 If $\theta$ is a solution of the gauge equation with connections $\left(\nabla_1, \nabla_2\right)$, then 
 $\theta^\star$ is a solution of the gauge equation with connections $\left(\nabla_2^\star, \nabla_1^\star\right)$
 \end{prop}
 \begin{proof}
 For $s \in \Gamma(E), \, \xi \in \Gamma(E^\star)$:
 \begin{align}
     \left(\nabla_2^\star (\theta^\star \xi)\right)\left(
     s\right) & = \left(\theta^\star \xi\right)(s) - 
     \left(\theta^\star \xi\right)\nabla_2 s 
     = \xi \left( \theta s \right) - \xi \left( \theta \nabla_2 s \right) \\
     & = \xi \left( \theta s \right) - \xi \left( \nabla_1 
     \theta s \right) 
      = \left(\theta^\star \nabla_1^\star \xi \right)(s)
 \end{align}
 \end{proof}
 Given a couple of connections $\left(\nabla_1 \nabla_2\right)$, the difference $D_{1,2} = \nabla_1 - \nabla_2$ is a section of $\Gamma\left( 
 TM^\star \otimes TM^\star \otimes E\right).$ Using it, the gauge equation \ref{def:gauge_equation} can rewritten as a commutation relation:
 \begin{equation}
     \label{eq:gauge_commutation}
     \nabla_2 \theta - \theta \nabla_2 + D_{1,2} \theta = 0
 \end{equation}
 Let $\tilde{E}$ be the bundle $E \oplus E^\star$. The bilinear form:
 \begin{equation}
     \label{eq:bilinear_etilde}
     B \colon \left(
     X+\alpha,Y+\beta
     \right) \in \tilde{E}^2 \to \beta(X)+\alpha(Y)
 \end{equation}
 is non-degenerate, namely:
 \begin{equation}
 \label{eq:non_degenerate_b}
 \forall Y+\beta \in \tilde{E} B\left( X + \alpha , Y + \beta \right) = 0 \Rightarrow X+\alpha = 0.
 \end{equation}
 \begin{prop}
 \label{prop:b_parallel}
 A $(1,1)$-tensor $\theta$ on $E$ satisfies the gauge equation for a couple of connections $\left( \nabla_1, \nabla_2\right)$ if and only the bilinear form:
 \begin{equation}
     \label{eq:bilinear_theta}
     B_\theta \colon \left(
     X+\alpha,Y+\beta
     \right) \to B\left(X + \alpha, \theta Y + \theta^\star 
     Y \right)
 \end{equation}
 is parallel with respect to the connection $\tilde{\nabla} = \nabla_2 \oplus \nabla_1^\star.$
 \end{prop}
 \begin{proof}
 By definition:
 \begin{equation}
  B\left(X + \alpha, \theta Y + \theta^\star 
     Y \right) = \alpha\left(
     \theta Y \right) + \theta^\star \beta \left( X
     \right)
 \end{equation}
 Taking the differential yields:
 \begin{align}
  d\left(\alpha( \theta Y) \right) & =
 \left(\nabla_1^\star\alpha \right)(\theta Y) +
 \alpha\left(
 \nabla_1 \theta Y
 \right) = \left(\theta^\star `\nabla_1\star \alpha \right) (Y) + 
 \alpha\left(
  \theta \nabla_2 Y
 \right) \\
 & = \left(\nabla_2^\star \theta^\star \alpha \right)(Y)  + 
 \alpha\left(
  \nabla_1 \theta Y
 \right)
 \end{align}
 and symmetrically:
 \begin{align}
   d\left((\theta^\star \beta)( X) \right) & = d\left(\beta)( \theta X) \right) \\
    & = \left(\nabla_2^\star \theta^\star \beta \right)(X)  + 
 \beta\left(
  \nabla_1 \theta X
 \right)
 \end{align}
 Now:
 \begin{align}
 &\tilde{B_\theta}\left(X+\alpha, Y + \beta\right)= \\
 &  
 d{B_\theta}\left(X+\alpha, Y + \beta\right) 
 -B_\theta\left(\tilde{\nabla}(X+\alpha),Y+\beta\right)
 -B_\theta\left(X+\alpha,\tilde{\nabla}(Y+\beta)\right) = \\
 & =  d{B_\theta}\left(X+\alpha, Y + \beta\right) 
 -\beta\left( \nabla_1 \theta X \right) +
 \left(\nabla_2^\star \theta^\star \alpha\right)(Y) 
  -\alpha\left( \nabla_1 \theta X \right) +
 \left(\nabla_2^\star \theta^\star\right) \beta(X) = 0
 \end{align}
 Conversely, if $B_\theta$ is $\tilde{\nabla}$-parallel, then, for any couple $(X+\alpha,Y+\beta)$:
 \begin{equation}
    \alpha \left( \left(\nabla_1 \theta - \theta \nabla_2\right)Y\right) + 
    \beta \left(\left(\nabla_1 \theta - \theta \nabla_2 \right)  X \right) = 0
 \end{equation}
 Taking, for example, $\alpha = 0$, $\beta$ arbitrary, it comes:
 \begin{equation}
     \left(\nabla_1 \theta - \theta \nabla_2 \right)  X = 0
 \end{equation}
 proving that the couple $\left( \theta_1, \theta_2 \right)$ satisfies the gauge equation.
 \end{proof}
 An immediate corollary is:
 \begin{cor}
 \label{cor:invariance_kernel}
 The kernel of $\tilde{\theta} = \theta \oplus \theta^*$ is 
 $\tilde{\nabla}$-invariant, hence the kernel of $\theta$ (resp. $\theta^\star$) is $\nabla_2$ (resp. $\nabla_1^\star$) invariant.
 \end{cor}
 \begin{proof}
 The kernel of $B$ is $\{0\}$, so if:
 \begin{equation}
     \forall Y + \beta \in \tilde{E}, \, B_\theta \left( X+\alpha, Y+\beta\right) = B(\theta X + \theta^\star \alpha, Y + \beta) = 0
 \end{equation}
 Then $\theta X + \theta^\star \alpha = 0$ and $X + \alpha \in \ker \tilde{\theta}.$ Given a basis of $\ker \tilde{\theta}$ at a point $p \in M$, parallel transporting it by $\tilde{\nabla}$ yields another basis of $\ker \tilde{\theta}$ at an arbitrary point $q\in M$, hence the claim.
 \end{proof}
 \begin{rem}
Corollary \ref{cor:invariance_kernel} implies by parallel transport that the dimension of the kernel of $\theta$ (resp. $\theta^\star$) is a constant, hence the rank of $\theta$ (resp. $\theta^\star$) is also a constant. 
 \end{rem}


 