\section{Introduction}
\subsection{Notations and writing conventions}
All manifolds are assumed to be smooth.
Thorough this document, the next writing conventions are applied:
$M$ is a smooth manifold. For a vector bundle  \begin{tikzcd}
    E \arrow[r,"\pi"] & M
\end{tikzcd}, the notation $\Gamma(U;E)$ with $U \subset M$ an open subset
of the manifold $M$ stands for the $C^\infty(M)$-module of smooth sections over
$U$. The functor $U \mapsto \Gamma(U;E)$ defines a sheaf denoted by $\Gamma_E$. Finally,
$\Gamma(E)$ is a shorthand notation for $\Gamma(M;E).$ Lowercase letters are used for sections,
uppercase ones for tangent vectors.
